% This line sets the project root file.
% !TEX root = Notes_Gauging_Defects.tex
% !TEX spellcheck = en_US

\subsection{Definition of the annular category}

%Let $\cat$ be a fusion category with objects $\mathrm{obj}(\cat)=\{X,Y,Z,...\}$, morphisms $\mathrm{hom}(X,Y)=\left\{f:X\rightarrow Y\vert X,Y\in\mathrm{obj}(\cat)\right\}$ and simple objects $\mathrm{Irr}(\cat)=\{A,B,C,...\}$. We denote an isomorphism class of objects in $\cat$ by $[U]$ and fix a representative $U_C$ in each isomorphism class, i.e. identify each isomorphism class with a simple object $C\in \mathrm{Irr}(\cat)$. The tube algebra $\tub$ is an algebra associated with $\cat$ given by the following construction
%\begin{align}
%	\tub&=\bigoplus_{A,B\in\mathrm{Irr}(\cat)}\,\tub_{BA}\\
%	\tub_{BA}&=\bigoplus_{C\in\mathrm{Irr}(\cat)}\, \mathrm{hom}(U_C\otimes U_A,U_B\otimes U_C).
%\end{align}
%\noindent
%The objects in $\tub$ are the same as the objects in $\cat$ where the morphisms are given by
%\begin{equation}
%	\mathrm{hom}(A,B)=X\otimes A\rightarrow B\otimes X.
%\end{equation}
%
The first step in our gauging process for a collection of bimodules $\{\mathcal{M}_i\}$ is to extend the fusion category $\cat$ to include the objects in $\mathcal{M}_i$. From Section~\ref{sec:bimodtensor}, we have fusion rules for bimodules $\mathcal{M}\otimes\mathcal{N}=\oplus\mathcal{P}$. We utilize the \emph{annular category} to obtain trivalent vertices for these processes. 

\begin{definition}[3-string annular category]
	Let $\mathcal{A},\mathcal{B},\mathcal{C}$ be fusion categories, and $\mathcal{A}\curvearrowright\mathcal{M}\curvearrowleft\mathcal{B}\curvearrowright\mathcal{N}\curvearrowleft\mathcal{C}$ and $\mathcal{A}\curvearrowright\mathcal{P}\curvearrowleft\mathcal{C}$ bimodule categories. The 3-string annular category $\ann_{\mathcal{M},\mathcal{N};\mathcal{P}}(\mathcal{A},\mathcal{B},\mathcal{C})$ is defined as follows:
	The simple objects are triples $(m,n;p)\in\mathcal{M}\times\mathcal{N}\times\mathcal{P}$. A basis for the morphism space $(m,n;p)\to (m^\prime,n^\prime;p^\prime)$ is given by valid diagrams on the annulus (up to isotopy and local relations)
	\begin{equation}
	\begin{tikzpicture}[scale=0.8,baseline=(current bounding box.center)]
	\def \rinner{.75};
	\def \router{2};
%	\node[red](m) at (0,0) {$*$};
%	\node(a) at (-1,0) {$a$};
%	\node(b) at (1,0) {$b$};
%	\node[red](abm) at (0,3) {$*$};
%	\draw[red] (m) -- (abm);
%	\draw (a) to [bend left] (0,1);
%	\draw (b) to [bend right] (0,2);
	\draw (0,0) circle (\rinner);
	\node[] at (90:\rinner-.25) {$p$};
	\node[] at (240:\rinner-.25) {$m$};
	\node[] at (300:\rinner-.25) {$n$};
	\draw (0,0) circle (\router);
	\node[] at (90:\router+.25) {$p^\prime$};
	\node[] at (240:\router+.3) {$m^\prime$};
	\node[] at (300:\router+.3) {$n^\prime$};
	\draw[blue] (90:\rinner)--(90:\router);
	\draw[red] (240:\rinner)--(240:\router);
	\draw[nicegreen] (300:\rinner)--(300:\router);
	\draw (240:1) to[out=90,in=240] (165:1) to [out=60,in=240] (90:1.25);
	\draw (300:1) to[out=90,in=300] (15:1) to [out=120,in=300] (90:1.5);
	\draw (240:1.75) to[out=40,in=240] (300:1.25) ;
	\node at (270:1.1) {$b$};
	\node at (180:1.25) {$a$};\node at (0:1.25) {$c$};
	\end{tikzpicture}.\label{eqn:basisann}
	\end{equation}
	Composition $Y\circ X$ corresponds to drawing the diagram $Y$ outside the diagram $X$ if the outer labels of $X$ match the inner labels of $Y$.
	The $F$ symbols of the fusion categories to reduce the composite diagram to a sum over diagrams of the form Eqn.~\ref{eqn:basisann}.
	
	$N$-string annular categories can be defined analogously. The 1-string annular category, with the string labeled by $\mathcal{C}$ itself, coincides with the \emph{tube algebra}\cite{ocneanu}. In the case that the single string is labeled by an invertible bimodule, this coincides with the \emph{dube algebra}\cite{WBV17}. 
\end{definition}