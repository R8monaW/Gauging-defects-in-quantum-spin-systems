% This line sets the project root file.
% !TEX root = Notes_Gauging_Defects.tex
% !TEX spellcheck = en_US

In the last few decades, several of the most important discoveries in condensed matter physics have happened in the field of topological states of matter (see, for example, \cite{Wen07,NSSFD08,HK10,QZ11}). For a long time it was thought that all continuous phase transitions could be described by the Ginzburg-Landau theory of symmetry-breaking, when in the late 1980s physicists discovered that some systems contain a new kind of order which is beyond the usual symmetry description. This proposed new kind of order, called \emph{topological order} \cite{Wen90}, soon became an interesting topic on its own, since it was found to be useful to describe different orders in different quantum Hall states \cite{WN90}. More recently, topologically ordered states are also considered for applications in topological quantum computing, because they have some remarkable properties: these quantum systems are formed from local interacting degrees of freedom which produce emergent global properties that are protected against environmental noise, which makes them a promising candidate for robust encoding, storage and manipulation of quantum systems \cite{DKLP2002,Kit03,NSSFD08,Ter15,PY15,BLPSW16}.

Although this is a very promising perspective, there are some drawbacks: in many phases, which includes those that are most suitable for experimental realization, the quantum computational power is severely limited. This has led to the study of defects in topologically ordered systems \cite{RH07,Bombin2010,KK12,FSV13,BJQ13b,BASP14,JPSV15,DIP16,CCW16,BBD17,CCW17b,CCW17,BLKW17,KPEB18,ET19}, since it soon became clear that a theory which includes those defects can describe topological phases with enhanced computational power, making them more suitable for quantum information processing tasks \cite{Freedman1998,FLW02b,FLW02,FKLW02}. For example, it is possible to realize a computationally universal braiding gate set by imposing defects to a topological state that itself is non-universal for topological quantum computation \cite{BJQ13}. 

In condensed matter physics, there is another approach to adding defects, which even in the case of simple and well-understood quantum systems substantially enriches their physics. Consider the simplest example of fermions moving freely in one spatial dimension: the addition of a single defect -- an \emph{impurity} -- leads to a substantially richer and more complex system exhibiting Kondo-type effects \cite{andersonLocalizedMagneticStates1961,hewsonKondoProblemHeavy1997}. The study of such impurity models by Wilson \cite{wilsonRenormalizationGroupCritical1975} directly led to the revolutionary development of the density matrix renormalization group (DMRG) \cite{whiteDensityMatrixFormulation1992}, and the subsequent tensor network revolution \cite{bridgemanHandwavingInterpretiveDance2017}.

Even in the absence of any symmetries, gapped quantum systems may form distinct phases of matter that exhibit topological order. However, in the presence of symmetries, the phases of these quantum systems have an even finer classification \cite{Wen2002,SRFL08,Kitaev2009,FK10,CGW11,FK11,TPB11,LS12,LV12,FM13,EH13,NCMT14,WPS14,K14,F14,EN14,MFCV15,BRSX15,LV16}. More precisely, it is possible for two quantum states that are equivalent when there is no symmetry to be distinct in the presence of symmetry. This phenomenon is referred to as symmetry-protected topological (SPT) states \cite{CGLW13,Yoshida2015,Yoshida2017} (if the gapped phase is trivial in the absence of any symmetry) or as symmetry-enriched topological (SET) states \cite{ENO10,MR13,WBV17} (if the phase is topologically nontrivial). As far as defects are concerned, not only the distinct phases of matter acquire a finer classification in the presence of symmetry, also the class of possible defects becomes larger. 

When studying a topological phase of matter that possesses a global symmetry, it is possible to consider the topological order that is obtained when the global symmetry is transformed into a local gauge invariance rather than the ungauged theory, which is referred to as ''gauging the symmetry``. This is convenient for several reasons: It allows, for instance, to study the properties of the ungauged systems by looking at the properties of the gauged theory \cite{LG12,HW12,Swingle2014,CG14}. Furthermore, it may give insights to the quantum phase transition between two different topological phases of matter \cite{BS09,BW10,BW11,BSS12}. 
The study of the fusion properties of defects in general quantum phases -- particularly when they are allowed to be mobile -- is deeply connected with global and local symmetries and the formalism of G-crossed braided tensor categories \cite{BBCW14}. This connection has been studied from a mathematical point of view in various works \cite{T00,ENO10,Turaev2010,CGPW16,EMJP18,CSZW18,D19,BJ19}. The study of models involving mobile defects -- analogues of impurity models for anyons -- with nontrivial dynamics lies at the very cutting edge of current research. Such models promise to provide new phases of matter even in $(1+1)$ dimensions, going beyond the SPT classification.

In our work, the goal is to make physical and microscopic sense of a dynamical theory of defects for quantum spin systems. While it is straightforward to define the kinematical data for a quantum system with an indefinite number of defects via the well-known Fock space construction, imposing dynamics is more complicated. In the context of topologically ordered systems, dynamical information is typically introduced via the ground eigenspace of a specific Hamiltonian that represents the topological properties of the system. A common example is Kitaev's toric code for $N\times N$ quantum spins arranged on the torus, which exhibits a four-dimensional ground eigenspace. Defects in this system could be modeled by absent quantum spins, for instance, which results in ground eigenspaces of different dimensions. When considering more that one defect (i.e., missing spin) one also has to distinguish between cases where the defects are next to each other and those where they are spatially separated. This can result in a quite complicated combinatorial problem, which we will investigate in more detail in Section \ref{Gauging}.

Furthermore, we study in great detail the one-dimensional example of a $\Vec(\Z/2\Z)$ spin chain, where defects are modeled by an invertible bimodule of the category in Section \ref{subsec:VecZ2}. Especially interesting and somewhat surprising is the fact that we find that the $F$-symbols of the gauged theory coincide exactly with those of the Ising category. Furthermore, defining a golden chain like Hamiltonian for this spin chain results in the transverse Ising model.