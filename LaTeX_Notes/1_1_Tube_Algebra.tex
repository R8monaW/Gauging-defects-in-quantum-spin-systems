% This line sets the project root file.
% !TEX root = Notes_Gauging_Defects.tex
% !TEX spellcheck = en_US

\subsection{Definition of the tube algebra}

%Let $\cat$ be a fusion category with objects $\mathrm{obj}(\cat)=\{X,Y,Z,...\}$, morphisms $\mathrm{hom}(X,Y)=\left\{f:X\rightarrow Y\vert X,Y\in\mathrm{obj}(\cat)\right\}$ and simple objects $\mathrm{Irr}(\cat)=\{A,B,C,...\}$. We denote an isomorphism class of objects in $\cat$ by $[U]$ and fix a representative $U_C$ in each isomorphism class, i.e. identify each isomorphism class with a simple object $C\in \mathrm{Irr}(\cat)$. The tube algebra $\tub$ is an algebra associated with $\cat$ given by the following construction
%\begin{align}
%	\tub&=\bigoplus_{A,B\in\mathrm{Irr}(\cat)}\,\tub_{BA}\\
%	\tub_{BA}&=\bigoplus_{C\in\mathrm{Irr}(\cat)}\, \mathrm{hom}(U_C\otimes U_A,U_B\otimes U_C).
%\end{align}
%\noindent
%The objects in $\tub$ are the same as the objects in $\cat$ where the morphisms are given by
%\begin{equation}
%	\mathrm{hom}(A,B)=X\otimes A\rightarrow B\otimes X.
%\end{equation}
%
%\begin{definition}[Fusion category]
%	A \emph{fusion category} consists of:
%	\begin{itemize}
%		\item A finite set of simple objects $L=\{1,a,b,c,\ldots\}$ , where $1$ is the distinguished unit object.
%		\item For each triple $(a,b;c)$ a finite dimensional Hilbert space $C(a,b;c)$ represented as a trivalent vertex
%		\begin{equation*}
%		\begin{tikzpicture}[scale=0.8,baseline=(current bounding box.center)]
%		\draw (-.707,-.707)--(0,0) node[pos=-.25] {$a$};
%		\draw (.707,-.707)--(0,0) node[pos=-.25] {$b$};
%		\draw (0,0)--(0,1) node[pos=1.25] {$c$};
%		\end{tikzpicture}
%		\end{equation*} 
%		\item A natural isomorphism  $F:(_\otimes_)
%	\end{itemize}
%	
%	In fact, the definition given here is a \emph{skeletal fusion category}.
%	In this manuscript, all constructions will be described using \emph{skeletal data}. This approach is standard in the (physics) literature, and amounts to picking representatives for each isomorphism class of simple object. The full fusion category can be canonically reconstructed from the skeletal data\cite{BBJSkeletal}.
%\end{definition}