\documentclass[a4paper,oneside]{amsart}

%\makeatletter
%\let\counterwithout\relax
%\makeatletter
%\let\counterwithin\relax

\usepackage{geometry}                % See geometry.pdf to 
\geometry{a4paper}                   % ... or a4paper or \usepackage{color}
\usepackage[utf8]{inputenc} %Sonderzeichen/Zeichenkodierung
\usepackage[english]{babel} %Sprache
\usepackage{graphicx}
\usepackage{amsmath, mathtools}
\usepackage{amsthm}
\usepackage{enumerate}
%\usepackage[shortalphabetic]{amsrefs}
\usepackage{wrapfig}
\usepackage[textsize=tiny]{todonotes}
\usepackage[many]{tcolorbox}
\usepackage{tensor} %Bimoduln
\usepackage{here} %Erzwingen von Grafikpositionen
\newcommand{\red}[1]{\textcolor{red}{#1}}
\definecolor{light-gray}{gray}{0.85}
\definecolor{nicegreen}{RGB}{60,183,82}
%\RequirePackage{doi}
%\usepackage{hyperref}
\usepackage{amssymb}

\usepackage[pdfpagelabels,pdftex,bookmarks,breaklinks]{hyperref}
\definecolor{darkblue}{RGB}{0,0,127} % choose colors
\definecolor{darkgreen}{RGB}{0,180,0}
\definecolor{darkred}{RGB}{180,0,0}
\hypersetup{
	colorlinks, 
	linkcolor=darkblue, 
	citecolor=darkgreen, 
	filecolor=red, 
	urlcolor=blue,
	pdftitle={Gauging defects in quantum spin systems}, 
	pdfauthor={Jacob C. Bridgeman, Alexander Hahn, Tobias J. Osborne, and Ramona Wolf}
}


\allowdisplaybreaks

%%Mathematik
%\usepackage{chemmacros}
%\usepackage{amssymb}
%\usepackage{latexsym}
%%Tabellen
%\usepackage{arydshln}
%\usepackage{multirow}
%\usepackage[automark]{scrpage2} %Kopf und Fu�zeilen
%\usepackage{rotating} %Rotieren von Grafiken
%\usepackage{braket} %Bra-Ket Notation
%\usepackage{blkarray} %Beschriftung von Matrizen
%\usepackage{mathrsfs} %curly Buchstaben

%tikz
\usepackage{tikz}
\usetikzlibrary{decorations.pathreplacing,shapes.misc}
\usetikzlibrary{arrows,shapes,positioning}
\usetikzlibrary{decorations.markings}
\usetikzlibrary{through,calc}
\usetikzlibrary{patterns}
\usetikzlibrary{plotmarks}
\usetikzlibrary{automata}
\tikzstyle arrowstyle=[scale=1]
\usepackage{extarrows}

%%% here we define our palette

\definecolor{spacecadet}{HTML}{0D284C}
\definecolor{munsell}{HTML}{008FA8}
\definecolor{banana}{HTML}{FFD932}
\definecolor{cgblue}{HTML}{007CA5}
\definecolor{isabelline}{HTML}{EAEDEA}

%%%

\DeclareMathOperator{\supp}{supp}
\DeclareMathOperator{\conf}{conf}
\DeclareMathOperator{\tr}{tr}
\DeclareMathOperator{\im}{im}
\DeclareMathOperator{\id}{id}
\DeclareMathOperator{\vspan}{span}
\DeclareMathOperator{\res}{res}
\DeclareMathOperator{\real}{Re}
\DeclareMathOperator{\imag}{Im}
\DeclareMathOperator{\diff}{diff}
\DeclareMathOperator{\homeo}{Homeo}
\DeclareMathOperator{\Mor}{Mor}
\DeclareMathOperator{\Ob}{Ob}
\DeclareMathOperator{\target}{Target}
\DeclareMathOperator{\source}{Source}
\DeclareMathOperator{\aut}{Aut}
\DeclareMathOperator{\Sch}{Sch}
\DeclareMathOperator{\Conf}{Conf}
\DeclareMathOperator{\Diff}{Diff}
\DeclareMathOperator{\Endstar}{End^*}
\DeclareMathOperator{\End}{End}

\newcommand{\cat}{\mathcal{C}}
\newcommand{\ann}{\mathrm{Ann}}
\renewcommand{\Vec}{\textbf{Vec}}
\newcommand{\Z}{\mathbb{Z}}

\theoremstyle{plain}% default
\newtheorem{theorem}{Theorem}[section]
\newtheorem{lemma}[theorem]{Lemma}
\newtheorem{proposition}[theorem]{Proposition}
\newtheorem*{corollary}{Corollary}

\theoremstyle{definition}
\newtheorem{definition}{Definition}[section]
\newtheorem{prototypedefinition}{Prototype Definition}[section]
\newtheorem{physicalintuition}{Physical intuition}[section]
\newtheorem{conjecture}{Conjecture}[section]
\newtheorem{example}{\color{cgblue}Example}[section]
\newtheorem{fact}{Fact}[section]

\theoremstyle{remark}
\newtheorem*{remark}{Remark}
\newtheorem*{note}{Note}

\usepackage[foot]{amsaddr}


\def\eqbox#1{\tcboxmath[colback=banana,arc=0cm,frame hidden,enhanced]{#1}}
\def\figbox{\begin{center}\tcboxmath[colback=isabelline,frame hidden,enhanced,sharp corners]{\parbox[c][3cm][c]{10cm}{\ }}\end{center}}