\documentclass[a4paper,10pt]{letter}
\usepackage[utf8]{inputenc}
\usepackage{comment}
\usepackage[pdftex]{graphicx,color}
\usepackage[dvipsnames]{xcolor}
\newcommand{\bl}[1]{{\color{blue}\it #1}}

\newcommand{\ra}[1]{{\color{red} #1}}
\newcommand{\rb}[1]{{\color{Green} #1}}
\newcommand{\rc}[1]{{\color{blue} #1}}
\newcommand{\rall}[1]{{\color{Orange} #1}}
\newcommand{\rgen}[1]{{\color{magenta} #1}}
\usepackage{braket}
\usepackage{mathtools}
\usepackage{adjustbox}
\usepackage[utf8]{inputenc}
%\usepackage{lineno}
%\usepackage[english]{babel}
\usepackage{csquotes}
\usepackage[dvipsnames]{xcolor}
\usepackage{color}
\usepackage{tikz}
\usepackage{graphics}
\usepackage{float}
\usepackage{todonotes}
\usetikzlibrary{shapes.geometric,backgrounds,positioning,shapes.geometric,decorations.markings,decorations.pathreplacing,arrows,knots,hobby,angles,quotes}
\hfuzz=2pt
\usepackage{amsmath}
\usepackage{amssymb}
\usepackage{amsthm}
\usepackage{braket}
\usepackage{bbm}

\DeclareMathOperator{\tr}{tr}
\begin{document}

\begin{letter}{TO_ADDRESS}

	
\begin{center}
REPLY TO REFEREES
\end{center}

\textbf{Authors}:\ We thank each of the referees for taking the time and dealing with our manuscript. We greatly appreciate the useful feedback and comments. In the following, we address each of the referees' points and describe the changes made to the manuscript based on their recommendations.

(The quoted text below from the referee is identical to the report, except that mathematics has been put into latex form.)
\vspace{0.5cm}

\begin{center}
	\textit{\textbf{Referee 1}}
\end{center}

1) I still think that the introductory sections IA and IB are 
unnecessarily lengthy, seeing that they do not contain many physical 
insights when taken on their own and merely serve as a motivation for 
what is analyzed in more detail in a more concrete setting in Sec. 
III. 

\begin{quote}
	\textbf{Authors}:\ 
\end{quote}

2) I am confused about the meaning of (and the difference between) the 
variables n and N in the introductory section, in particular in Eq. 
(5) and the discussion thereof. I wish the authors could clarify this. 

\begin{quote}
	\textbf{Authors}:\ Thank you for bringing our attention out this point which is indeed confusing. We have revised the formulas and made the choice of variables consistent.
\end{quote}

3) The following statement in the beginning of Sec. II is confusing 
with regard to the structure of the paper: 

``We will first have to specify the kinematics of quantum spin chains 
with defects. Afterwards, we will allow the defects to become mobile 
by the action of a local Hamiltonian. By investigating the simple 
example of Kitaev's toric code [8] we will show that this is an 
intricate combinatorial problem, which motivates to make use of 
alternative ways to describe kinematics.''

This makes it sound as if a discussion of the Toric code was still to 
follow, which it is not. Indeed, the Toric code was only mentioned in 
Sec. I as an introductory example. 

\begin{quote}
	\textbf{Authors}:\ We thank the referee for pointing this out and have changed this paragraph to the following:
		
	``The investigation of the simple example of Kitaev's toric code in the previous section has shown that describing a lattice system with defects is an intricate combinatorial problem, which motivates the use of alternative ways to describe kinematics.''
\end{quote}

4) In passing, let me also note that there seems to be a typo in the 
line following Eq. (10), where I assume the last subscript of 
$\mathcal{V}$ should read $e_{N^2}$ instead of $e_N^2$.

\begin{quote}
	\textbf{Authors}:\ Thanks for spotting this error, we have corrected that.
\end{quote}


\begin{center}
	\textit{\textbf{Referee 2}}
\end{center}

1) I urge the authors to go through the paper carefully and weed out 
typos. For example, there seems to be confusion between $n$ and $N$ in the 
Introduction. In my opinion $N$ should be replaced by $n$ in the right 
hand side of Eqs. 5,6,7.

\begin{quote}
	\textbf{Authors}:\ We thank the referee for mentioning this point. The choice of variable is indeed inconsistent which we have revised and changed in the manuscript.
\end{quote}

\end{letter}

\end{document}


