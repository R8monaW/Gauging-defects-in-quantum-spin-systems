\documentclass[a4paper,10pt]{letter}
\usepackage[utf8]{inputenc}
\usepackage{comment}
\usepackage[pdftex]{graphicx,color}
\usepackage[dvipsnames]{xcolor}
\newcommand{\bl}[1]{{\color{blue}\it #1}}

\newcommand{\ra}[1]{{\color{red} #1}}
\newcommand{\rb}[1]{{\color{Green} #1}}
\newcommand{\rc}[1]{{\color{blue} #1}}
\newcommand{\rall}[1]{{\color{Orange} #1}}
\newcommand{\rgen}[1]{{\color{magenta} #1}}
\usepackage{braket}
\usepackage{mathtools}
\usepackage{adjustbox}
\usepackage[utf8]{inputenc}
%\usepackage{lineno}
%\usepackage[english]{babel}
\usepackage{csquotes}
\usepackage[dvipsnames]{xcolor}
\usepackage{color}
\usepackage{tikz}
\usepackage{graphics}
\usepackage{float}
\usepackage{todonotes}
\usetikzlibrary{shapes.geometric,backgrounds,positioning,shapes.geometric,decorations.markings,decorations.pathreplacing,arrows,knots,hobby,angles,quotes}
\hfuzz=2pt
\usepackage{amsmath}
\usepackage{amssymb}
\usepackage{amsthm}
\usepackage{braket}
\usepackage{bbm}

\DeclareMathOperator{\tr}{tr}
\begin{document}

\begin{letter}{TO_ADDRESS}

	
\begin{center}
REPLY TO REFEREE
\end{center}

\textbf{Authors}:\ We are very thankful to the referee for taking the time to read our manuscript and give helpful feedback. We appreciate the valuable comments and suggestions which helped us to improve the comprehensibility and clarity of the content. In the following, we address each of the referee's points and describe the changes made to the manuscript based on the recommendations.

(The quoted text below from the referee is identical to the report.)
\vspace{0.5cm}

1) The introduction should be partly rewritten in order to correctly reflect the fact that the bulk of the paper presents a case study of a particular spin system with defects, not a general theory. 

\begin{quote}
	\textbf{Authors}:\ We thank the referee for pointing out that the title, abstract and introduction could be somewhat misleading. We have dealed with this problem by making the following changes:
		\begin{enumerate}
			\item We changed the title of the paper to ``Gauging defects in quantum spin systems: a case study'' to emphasize that the main part of our paper is a detailed study of the $\mathbf{Vec}(\mathbb{Z}/2\mathbb{Z})$ case.
			\item We have slightly rewritten the abstract of the paper to make clear that the focus is on this case study.
			\item We have partly rewritten and expanded the introduction (see also next point), such that it becomes clear that we focus on a specific case in this paper.
		\end{enumerate}
\end{quote}

2) Section II does not contain any deep results when taken on its own, and therefore should be more clearly presented as introductory chapter leading up to Section IV. In its present form, Section II seems somewhat disconnected from the rest of the paper.

\begin{quote}
	\textbf{Authors}:\ We thank the referee for bringing this to our attention. Indeed, Section II explains the physical ideas and constructions in the study of dynamical defects and motivates the use of category theory in order to deal with the highly intricately combinatorial problem, which we then demonstrate explicitly in the case study in Section IV. We have now included most of this chapter in the introduction such that the introduction motivates and leads to the case study. The last part of which was previously Section II is now included in the case study in Section IV so that this is nicely connected to the content of the introduction.
\end{quote}


\end{letter}

\end{document}


