% This line sets the project root file.
% !TEX root = Notes_Gauging_Defects.tex
% !TEX spellcheck = en_US

Many important discoveries in condensed matter physics during the past decades have arisen from the study of \emph{topological states of matter} (see, for example, \cite{Wen07,NSSFD08,HK10,QZ11}). For a long time it was thought that all continuous phase transitions could be described by the Ginzburg-Landau theory of symmetry breaking. In the late 1980s, however, physicists discovered that some systems can exhibit a new kind of order going beyond the usual symmetry breaking paradigm. This new kind of order -- \emph{topological order} \cite{Wen90} -- soon became an interesting topic in its own right, useful for the description of different quantum Hall states \cite{WN90}. More recently, topologically ordered systems gave rise to \emph{topological quantum computation}, because of their remarkable properties: these locally interacting systems exhibit emergent global properties protected against environmental noise. This makes them a promising platform for the robust encoding, storage, and manipulation of quantum systems \cite{DKLP2002,Kit03,NSSFD08,Ter15,PY15,BLPSW16}.

However, in many phases, which includes those most suitable for experimental realization, the quantum computational power of a topologically ordered system is severely limited. To ameliorate this, defects in topologically ordered systems were considered \cite{RH07,Bombin2010,KK12,FSV13,BJQ13b,BASP14,JPSV15,DIP16,CCW16,BBD17,CCW17b,CCW17,BLKW17,KPEB18,ET19}, since a theory which includes defects can describe topological phases with enhanced computational power \cite{Freedman1998,FLW02b,FLW02,FKLW02}. For example, it is possible to use topologically ordered systems which are not universal for quantum computation to realize a computationally universal braiding gate set by adding defects \cite{BJQ13}.

In condensed matter physics, there is another approach to adding defects, which even in the case of simple and well-understood quantum systems substantially enriches their physics. Consider a simple example of fermions moving freely in one spatial dimension: the addition of a single defect -- an \emph{impurity} -- leads to a substantially richer and more complex system exhibiting Kondo-type effects \cite{andersonLocalizedMagneticStates1961,hewsonKondoProblemHeavy1997}. The study of such impurity models by Wilson \cite{wilsonRenormalizationGroupCritical1975} directly led to the revolutionary development of the density matrix renormalization group (DMRG) \cite{whiteDensityMatrixFormulation1992}, and the subsequent tensor network revolution \cite{bridgemanHandwavingInterpretiveDance2017}.

In the presence of additional symmetries, the phases of quantum systems have an even finer classification \cite{Wen2002,SRFL08,Kitaev2009,FK10,CGW11,FK11,TPB11,LS12,LV12,FM13,EH13,NCMT14,WPS14,K14,F14,EN14,MFCV15,BRSX15,LV16}. More precisely, it is possible for two quantum states that are equivalent when there is no additional symmetry present to be distinct in the presence of additional symmetries. This phenomenon is referred to as \emph{symmetry-protected topological} order (SPT) \cite{CGLW13,Yoshida2015,Yoshida2017} if the gapped phase is trivial in the absence of any symmetry, or as \emph{symmetry-enriched topological} (SET) order \cite{ENO10,MR13,WBV17} if the phase is topologically nontrivial. As far as defects are concerned, not only the distinct phases of matter acquire a finer classification in the presence of additional symmetries, but also the class of possible defects becomes larger. 

It is possible to to transform a topologically ordered system in the presence of a global symmetry to a system with a local gauge invariance. This is referred to as ``gauging the symmetry''. This is convenient for several reasons. For example, many properties of the ungauged system can be understood by looking at the properties of the gauged theory \cite{LG12,HW12,Swingle2014,CG14}. Furthermore, it may give insights into the quantum phase transitions between two different topological phases of matter \cite{BS09,BW10,BW11,BSS12}. 
The study of the fusion properties of defects in general quantum phases -- particularly when they are allowed to be mobile -- is deeply connected with global and local symmetries and the formalism of \emph{G-crossed braided tensor categories} \cite{BBCW14}. This connection has been studied from a mathematical point of view in various works \cite{T00,ENO10,Turaev2010,CGPW16,EMJP18,CSZW18,D19,BJ19}. The study of models involving mobile defects -- analogues of impurity models for anyons -- with nontrivial dynamics lies at the very cutting edge of current research. Such models promise to provide new phases of matter even in $(1+1)$ dimensions, going beyond the SPT classification.

In our work, the goal is to make physical and microscopic sense of a dynamical theory of defects for a one-dimensional quantum spin chain with $\Vec(\Z/2\Z)$ fusion rules. While we are focusing on this particular case study, the framework we establish can easily be extended to arbitrary quantum spin chains. Define the kinematical data for the system with an indefinite number of defects is straightforward via the well-known Fock space construction, whereas imposing dynamics is more complicated. In the context of topologically ordered systems, such as the considered $\Vec(\Z/2\Z)$ spin chain, dynamical information is typically introduced via the ground eigenspace of a specific Hamiltonian that represents the topological properties of the system. A common example is Kitaev's toric code for $N\times N$ quantum spins arranged on the torus, which exhibits a four-dimensional ground eigenspace. Defects in this system could be modeled by absent quantum spins, resulting in ground eigenspaces of differing dimensions. When considering more than one defect (i.e., missing spins) one also has to distinguish between cases where the defects are next to each other and those where they are spatially separated. This can result in a complicated combinatorial problem as shown in Subsec.~\ref{Kin_Dyn}. Therefore, we use tools from category theory to simplify explicit computations.

In our particular example of the one-dimensional example of a $\Vec(\Z/2\Z)$ spin chain defects are modeled by an invertible bimodule of the underlying category, see Subsec.~\ref{subsec:VecZ2}.
Using generalized tube algebra techniques \cite{ocneanu}, we construct explicit trivalent vertices for the gauged theory. Consistent with the literature \cite{TY,ENO10,Bombin2010,BBCW14,WBV17}, we find that the $F$-symbols of the gauged theory (computed using these explicit vertices) coincide exactly with those of the Ising category. To the best of our knowledge, this is the first time such a result has been computed directly using the tube algebra. Furthermore, defining a golden-chain-like Hamiltonian \cite{Feiguin2007} for this spin chain results in the transverse Ising model.

This paper is structured as follows: In Sec.~\ref{S:defs}, we provide definitions and background required for the remainder of the paper. We start by showing how to construct the Hilbert space for a quantum spin system in the presence of defects. This is followed by a recap on how to incorporate dynamics to these systems in the presence of topological order. By briefly studying the simple example of Kitaev's toric code it becomes clear that this task ends up in an extremely cumbersome combinatorial exercise. That is why we will need to introduce the language of category theory to continue our studies. In particular, we need the concepts from generalized tube algebras provided here to be able to study explicit examples of models with dynamical defects. In Sec.~\ref{Ising}, we transfer the previously introduced concepts to a spin chain with $\Vec(\Z/2\Z)$ fusion rules, which enables us to compute its dynamical data. This leads to a realization of the transverse Ising model. Finally, we conclude in Sec.~\ref{Conclusion}.