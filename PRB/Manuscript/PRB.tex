%% ****** Start of file template.aps ****** %
%%
%%
%%   This file is part of the APS files in the REVTeX 4 distribution.
%%   Version 4.0 of REVTeX, August 2001
%%
%%
%%   Copyright (c) 2001 The American Physical Society.
%%
%%   See the REVTeX 4 README file for restrictions and more information.
%%
%
% This is a template for producing manuscripts for use with REVTEX 4.0
% Copy this file to another name and then work on that file.
% That way, you always have this original template file to use.
%
% Group addresses by affiliation; use superscriptaddress for long
% author lists, or if there are many overlapping affiliations.
% For Phys. Rev. appearance, change preprint to twocolumn.
% Choose pra, prb, prc, prd, pre, prl, prstab, or rmp for journal
%  Add 'draft' option to mark overfull boxes with black boxes
%  Add 'showpacs' option to make PACS codes appear
\documentclass[aps,prb,twocolumn,superscriptaddress,noshowkeys]{revtex4-1}  % for review and submission
%\documentclass[aps,preprint,showpacs,superscriptaddress,groupedaddress]{revtex4}  % for double-spaced preprint
\usepackage{graphicx}  % needed for figures
\usepackage{dcolumn}   % needed for some tables
\usepackage{bm}        % for math
\usepackage{amssymb}   % for math

%\usepackage{geometry}                % See geometry.pdf to 
%\geometry{a4paper}                   % ... or a4paper or \usepackage{color}
\usepackage[utf8]{inputenc} %Sonderzeichen/Zeichenkodierung
\usepackage[english]{babel} %Sprache
\usepackage{amsmath, mathtools}
\usepackage{amsthm}
\usepackage{enumerate}
\usepackage{wrapfig}
\usepackage[textsize=tiny]{todonotes}
\usepackage[many]{tcolorbox}
\usepackage{tensor} %Bimoduln
\usepackage{here} %Erzwingen von Grafikpositionen
\usepackage{subcaption}%subfigure
\captionsetup{compatibility=false}
\newcommand{\red}[1]{\textcolor{red}{#1}}
\definecolor{light-gray}{gray}{0.85}
\definecolor{nicegreen}{RGB}{60,183,82}
\usepackage[pdfpagelabels,pdftex,bookmarks,breaklinks]{hyperref}
\definecolor{PRBblue}{RGB}{0,0,254}
\definecolor{bwred}{rgb}{0.94,0.5,0.5}
\definecolor{bwblue}{rgb}{0.13,0.67,0.8}
\definecolor{atomictangerine}{rgb}{1.0, 0.6, 0.4}
%\definecolor{darkblue}{RGB}{0,0,127} % choose colors
%\definecolor{darkgreen}{RGB}{0,180,0}
%\definecolor{darkred}{RGB}{180,0,0}
\hypersetup{
	colorlinks, 
	linkcolor=PRBblue, 
	citecolor=PRBblue, 
	filecolor=PRBblue, 
	urlcolor=PRBblue,
	pdftitle={Gauging defects in quantum spin systems}, 
	pdfauthor={Jacob C. Bridgeman, Alexander Hahn, Tobias J. Osborne, and Ramona Wolf}
}
\usepackage{stackengine}

\allowdisplaybreaks

%tikz
\usepackage{tikz}
\usetikzlibrary{decorations.pathreplacing,shapes.misc}
\usetikzlibrary{arrows,shapes,positioning}
\usetikzlibrary{decorations.markings}
\usetikzlibrary{through,calc}
\usetikzlibrary{patterns}
\usetikzlibrary{plotmarks}
\usetikzlibrary{automata}
\tikzstyle arrowstyle=[scale=1]
\usepackage{extarrows}

%%%

\DeclareMathOperator{\supp}{supp}
\DeclareMathOperator{\conf}{conf}
\DeclareMathOperator{\tr}{tr}
\DeclareMathOperator{\im}{im}
\DeclareMathOperator{\id}{id}
\DeclareMathOperator{\vspan}{span}
\DeclareMathOperator{\res}{res}
\DeclareMathOperator{\real}{Re}
\DeclareMathOperator{\imag}{Im}
\DeclareMathOperator{\diff}{diff}
\DeclareMathOperator{\homeo}{Homeo}
\DeclareMathOperator{\Mor}{Mor}
\DeclareMathOperator{\Ob}{Ob}
\DeclareMathOperator{\target}{Target}
\DeclareMathOperator{\source}{Source}
\DeclareMathOperator{\aut}{Aut}
\DeclareMathOperator{\Sch}{Sch}
\DeclareMathOperator{\Conf}{Conf}
\DeclareMathOperator{\Diff}{Diff}
\DeclareMathOperator{\Endstar}{End^*}
\DeclareMathOperator{\End}{End}

\newcommand{\cat}{\mathcal{C}}
\newcommand{\ann}{\mathrm{Ann}}
\renewcommand{\Vec}{\textbf{Vec}}
\newcommand{\Z}{\mathbb{Z}}

\theoremstyle{plain}% default
\newtheorem{theorem}{Theorem}[section]
\newtheorem{lemma}[theorem]{Lemma}
\newtheorem{proposition}[theorem]{Proposition}
\newtheorem*{corollary}{Corollary}

\theoremstyle{definition}
\newtheorem{definition}{Definition}[section]
\newtheorem{prototypedefinition}{Prototype Definition}[section]
\newtheorem{physicalintuition}{Physical intuition}[section]
\newtheorem{conjecture}{Conjecture}[section]
\newtheorem{example}{Example}[section]
\newtheorem{fact}{Fact}[section]

\theoremstyle{remark}
\newtheorem*{remark}{Remark}
\newtheorem*{note}{Note}

% avoids incorrect hyphenation, added Nov/08 by SSR
\hyphenation{ALPGEN}
\hyphenation{EVTGEN}
\hyphenation{PYTHIA}

%Section numbering
%\renewcommand{\thesection}{\Roman{section}}
%\setcounter{secnumdepth}{1}
%\renewcommand{\thesubsection}{\Alph{subsection}}
%\setcounter{secnumdepth}{2}

\begin{document}

% the following line is for submission, including submission to the arXiv!!
%\hspace{5.2in} \mbox{Fermilab-Pub-04/xxx-E}

\title{Gauging defects in quantum spin systems: a case study}

%\author{
%	Jacob C. Bridgeman,$^1$ Alexander Hahn,$^2$ Tobias J. Osborne,$^2$ and Ramona Wolf$^2$\\
%	$^{1}$\it{Perimeter Institute for Theoretical Physics, 31 Caroline St N, Waterloo, ON N2L 2Y5, Canada}\\
%	$^{2}$\it{Institut für Theoretische Physik, Leibniz Universität Hannover, Appelstraße 2, 30167 Hannover, Germany}
%}

\author{Jacob C. Bridgeman}
\email{jcbridgeman1@gmail.com}
\affiliation{Perimeter Institute for Theoretical Physics, 31 Caroline St N, Waterloo, ON N2L 2Y5, Canada}
\author{Alexander Hahn}
\email{alexander.hahn@htp-tel.de}
\affiliation{Institut für Theoretische Physik, Leibniz Universität Hannover, Appelstraße 2, 30167 Hannover, Germany}
\author{Tobias J. Osborne}
\email{tobias.osborne@itp.uni-hannover.de}
\affiliation{Institut für Theoretische Physik, Leibniz Universität Hannover, Appelstraße 2, 30167 Hannover, Germany}
\author{Ramona Wolf}
\email{ramona.wolf@itp.uni-hannover.de}
\affiliation{Institut für Theoretische Physik, Leibniz Universität Hannover, Appelstraße 2, 30167 Hannover, Germany}

                            

\begin{abstract}
% !TEX root = Notes_Gauging_Defects.tex
The goal of this work is to build a dynamical theory of defects for quantum spin systems. A kinematic theory for an indefinite number of defects is first introduced exploiting \emph{distinguishable Fock space}. Dynamics are then incorporated by allowing the defects to become mobile via a microscopic Hamiltonian. This construction is extended to topologically ordered systems by restricting to the ground state eigenspace of Hamiltonians generalizing the \emph{golden chain}. We illustrate the construction with the example of a spin chain with $\Vec(\Z/2\Z)$ fusion rules, employing generalized tube algebra techniques to model the defects in the chain. The resulting dynamical defect model is equivalent to the critical transverse Ising model.
\end{abstract}


\maketitle

\section{Introduction}
% This line sets the project root file.
% !TEX root = Notes_Gauging_Defects.tex
% !TEX spellcheck = en_US

Many important discoveries in condensed matter physics during the past decades have arisen from the study of \emph{topological states of matter} (see, for example, \cite{Wen07,NSSFD08,HK10,QZ11}). For a long time it was thought that all continuous phase transitions could be described by the Ginzburg-Landau theory of symmetry breaking. In the late 1980s, however, physicists discovered that some systems can exhibit a new kind of order going beyond the usual symmetry breaking paradigm. This new kind of order -- \emph{topological order} \cite{Wen90} -- soon became an interesting topic in its own right, useful for the description of different quantum Hall states \cite{WN90}. More recently, topologically ordered systems gave rise to \emph{topological quantum computation}, because of their remarkable properties: these locally interacting systems exhibit emergent global properties protected against environmental noise. This makes them a promising platform for the robust encoding, storage, and manipulation of quantum systems \cite{DKLP2002,Kit03,NSSFD08,Ter15,PY15,BLPSW16}.

However, in many phases, which includes those most suitable for experimental realization, the quantum computational power of a topologically ordered system is severely limited. To ameliorate this, defects in topologically ordered systems were considered \cite{RH07,Bombin2010,KK12,FSV13,BJQ13b,BASP14,JPSV15,DIP16,CCW16,BBD17,CCW17b,CCW17,BLKW17,KPEB18,ET19}, since a theory which includes defects can describe topological phases with enhanced computational power \cite{Freedman1998,FLW02b,FLW02,FKLW02}. For example, it is possible to use topologically ordered systems which are not universal for quantum computation to realize a computationally universal braiding gate set by adding defects \cite{BJQ13}.

In condensed matter physics, there is another approach to adding defects, which even in the case of simple and well-understood quantum systems substantially enriches their physics. Consider a simple example of fermions moving freely in one spatial dimension: the addition of a single defect -- an \emph{impurity} -- leads to a substantially richer and more complex system exhibiting Kondo-type effects \cite{andersonLocalizedMagneticStates1961,hewsonKondoProblemHeavy1997}. The study of such impurity models by Wilson \cite{wilsonRenormalizationGroupCritical1975} directly led to the revolutionary development of the density matrix renormalization group (DMRG) \cite{whiteDensityMatrixFormulation1992}, and the subsequent tensor network revolution \cite{bridgemanHandwavingInterpretiveDance2017}.

In the presence of additional symmetries, the phases of quantum systems have an even finer classification \cite{Wen2002,SRFL08,Kitaev2009,FK10,CGW11,FK11,TPB11,LS12,LV12,FM13,EH13,NCMT14,WPS14,K14,F14,EN14,MFCV15,BRSX15,LV16}. More precisely, it is possible for two quantum states that are equivalent when there is no additional symmetry present to be distinct in the presence of additional symmetries. This phenomenon is referred to as \emph{symmetry-protected topological} order (SPT) \cite{CGLW13,Yoshida2015,Yoshida2017} if the gapped phase is trivial in the absence of any symmetry, or as \emph{symmetry-enriched topological} (SET) order \cite{ENO10,MR13,WBV17} if the phase is topologically nontrivial. As far as defects are concerned, not only the distinct phases of matter acquire a finer classification in the presence of additional symmetries, but also the class of possible defects becomes larger. 

It is possible to to transform a topologically ordered system in the presence of a global symmetry to a system with a local gauge invariance. This is referred to as ``gauging the symmetry''. This is convenient for several reasons. For example, many properties of the ungauged system can be understood by looking at the properties of the gauged theory \cite{LG12,HW12,Swingle2014,CG14}. Furthermore, it may give insights into the quantum phase transitions between two different topological phases of matter \cite{BS09,BW10,BW11,BSS12}. 
The study of the fusion properties of defects in general quantum phases -- particularly when they are allowed to be mobile -- is deeply connected with global and local symmetries and the formalism of \emph{G-crossed braided tensor categories} \cite{BBCW14}. This connection has been studied from a mathematical point of view in various works \cite{T00,ENO10,Turaev2010,CGPW16,EMJP18,CSZW18,D19,BJ19}. The study of models involving mobile defects -- analogues of impurity models for anyons -- with nontrivial dynamics lies at the very cutting edge of current research. Such models promise to provide new phases of matter even in $(1+1)$ dimensions, going beyond the SPT classification.

In our work, the goal is to make physical and microscopic sense of a dynamical theory of defects for a one-dimensional quantum spin chain with $\Vec(\Z/2\Z)$ fusion rules. While we are focusing on this particular case study, the framework we establish can easily be extended to arbitrary quantum spin chains. Define the kinematical data for the system with an indefinite number of defects is straightforward via the well-known Fock space construction, whereas imposing dynamics is more complicated. In the context of topologically ordered systems, such as the considered $\Vec(\Z/2\Z)$ spin chain, dynamical information is typically introduced via the ground eigenspace of a specific Hamiltonian that represents the topological properties of the system. A common example is Kitaev's toric code for $N\times N$ quantum spins arranged on the torus, which exhibits a four-dimensional ground eigenspace. Defects in this system could be modeled by absent quantum spins, resulting in ground eigenspaces of differing dimensions. When considering more than one defect (i.e., missing spins) one also has to distinguish between cases where the defects are next to each other and those where they are spatially separated. This can result in a complicated combinatorial problem as shown in Subsec.~\ref{Kin_Dyn}. Therefore, we use tools from category theory to simplify explicit computations.

In our particular example of the one-dimensional example of a $\Vec(\Z/2\Z)$ spin chain defects are modeled by an invertible bimodule of the underlying category, see Subsec.~\ref{subsec:VecZ2}.
Using generalized tube algebra techniques \cite{ocneanu}, we construct explicit trivalent vertices for the gauged theory. Consistent with the literature \cite{TY,ENO10,Bombin2010,BBCW14,WBV17}, we find that the $F$-symbols of the gauged theory (computed using these explicit vertices) coincide exactly with those of the Ising category. To the best of our knowledge, this is the first time such a result has been computed directly using the tube algebra. Furthermore, defining a golden-chain-like Hamiltonian \cite{Feiguin2007} for this spin chain results in the transverse Ising model.

This paper is structured as follows: In Sec.~\ref{S:defs}, we provide definitions and background required for the remainder of the paper. We start by showing how to construct the Hilbert space for a quantum spin system in the presence of defects. This is followed by a recap on how to incorporate dynamics to these systems in the presence of topological order. By briefly studying the simple example of Kitaev's toric code it becomes clear that this task ends up in an extremely cumbersome combinatorial exercise. That is why we will need to introduce the language of category theory to continue our studies. In particular, we need the concepts from generalized tube algebras provided here to be able to study explicit examples of models with dynamical defects. In Sec.~\ref{Ising}, we transfer the previously introduced concepts to a spin chain with $\Vec(\Z/2\Z)$ fusion rules, which enables us to compute its dynamical data. This leads to a realization of the transverse Ising model. Finally, we conclude in Sec.~\ref{Conclusion}.

\section{Background}\label{S:defs}
In this section, we provide all the definitions and notation that are used throughout the subsequent computations. We will first have to specify the kinematics of quantum spin chains with defects. Afterwards, we will allow the defects to become mobile by the action of a local Hamiltonian. By investigating the simple example of Kitaev's toric code \cite{Kit03} we will show that this is an intricate combinatorial problem, which motivates to make use of alternative ways to describe kinematics. We will therefore use the language of category theory in order to overcome complications. We will familiarize the reader with this language by not giving exhaustive details and explanations whenever it is not necessary for this work but instead referring to the literature where those can be found.

%\subsection{Kinematics and dynamics for quantum spin chains with defects}\label{Kin_Dyn}

Our goal in this introductory subsection is to present a framework to make physical and microscopic sense of a dynamical theory of defects for quantum (spin) systems. We restrict our attention to defects modeled by absent/missing quantum spins. (Extending this idea to more general defects modeled by different Hilbert spaces then becomes a straightforward task.)

Firstly, we focus on the problem of modeling an \emph{indefinite} number of distinguishable spins. Building on this we can propose a kinematical space to model arbitrary numbers of quantum spins at various locations. This allows us to then consider the case of missing-spin defects for topologically ordered systems.

When we want to describe a quantum system with an \emph{indefinite} number of \emph{indistinguishable} particles we should use \emph{Fock space}. We will use the arguably conceptually simpler construction of \emph{distinguishable Fock space} (see \cite{OsborneVideoLectureAQT} for a course where this is explained), appropriate for describing a collection of an indefinite number of distinguishable particles. Subsequently, one can obtain Bose-/Fermi-Fock space by imposing an equivalence relation under particle exchange. Distinguishable Fock space is given by
\begin{equation}
\mathfrak{F}(\mathbb{C}^d) \cong \bigoplus_{N=0}^\infty \mathcal{H}_N\cong \bigoplus_{N=0}^\infty (\underbrace{\mathbb{C}^d\otimes \mathbb{C}^d\otimes \cdots \otimes \mathbb{C}^d}_{\text{$N$ factors}}),
\end{equation}
where $\mathcal{H}_N$ is the Hilbert space for $N$ distinguishable particles. By convention and assumption the space describing zero particles, the \emph{vacuum}, is assigned the Hilbert space $(\mathbb{C}^d)^{\otimes 0} \cong \mathbb{C}$.

It is now easy to incorporate additional constraints on the numbers of particles, e.g., to describe a system comprised of \emph{either} zero distinguishable quantum spins \emph{or} one distinguishable quantum spin we would use
\begin{equation}
\mathfrak{F}_{\le 1} (\mathbb{C}^d) \cong \mathbb{C}\oplus \mathbb{C}^d.
\end{equation}
We could call this the Hilbert space of a ``maybe'' quantum spin.

Let's now consider the central situation for this work. How should we describe a quantum lattice of $n$ sites, where either a single quantum spin is present at a site, or not? (We call the case where a spin is absent a \emph{defect}.) According to the discussion above we should simply tensor up $N$ ``maybe'' quantum spins:
\begin{equation}
\mathfrak{F}_{\le n}(\mathbb{C}^d) \equiv \bigotimes_{N=0}^n \mathfrak{F}_{\le 1}(\mathbb{C}^d) \cong (\mathbb{C}\oplus \mathbb{C}^d)^{\otimes N}.
\end{equation}
Expanding out the tensor factors leads to the equivalent definition
\begin{equation}
\mathfrak{F}_{\le n}(\mathbb{C}^d) \equiv \bigoplus_{j=0}^N \mathbb{C}^{\binom{N}{j}}\otimes \mathcal{H}_j.
\end{equation}
The Hilbert space $\mathfrak{F}_{\le n}(\mathbb{C}^d)$ supplies us with just the kinematical data to describe a system of distinguishable particles. To incorporate additional dynamical information we must specify additional \emph{dynamical} data.

In the context of topologically ordered models such as Kitaev's toric code \cite{Kit03} we typically introduce dynamical information indirectly by describing the system via the ground eigenspace $\mathcal{V}\subset \mathcal{H}$ of a specific Hamiltonian $H$. In the case of the toric code for $N\times N$ quantum spins arranged on the torus, the ground eigenspace $\mathcal{V}_{\mathbb{T}}$ of the toric code Hamiltonian is four dimensional. Let us now assume we have a lattice with a missing-qubit defect at some edge $e$ \cite{BLKW17}, see Fig. \ref{fig:1defect}. One can define a toric-code Hamiltonian for this new punctured torus; restricting to its ground eigenspace yields a \emph{four-dimensional subspace} $\mathcal{V}_{\mathbb{T}\setminus e}$. There is no obstruction to describing the ground eigenspace for two, three, etc.\ missing-qubit defects. What results is a rather intricate combinatorial problem, as, depending on where the defects are located relative to each other, one gets higher or lower dimensional ground eigenspace. To see an example of the intricacies that easily result, consider the case of two missing edges: these can either be adjacent to each other or distant. In the latter case, depicted in Fig. \ref{fig:2distdefects}, the ground space for this system is $(\mathbb{C}^2)^{\otimes 3}$ while in the case of adjacent defects we have a single larger puncture (with smooth boundary), see Fig. \ref{fig:2adjdefects}, and the ground eigenspace is only $(\mathbb{C}^2)^{\otimes 2}$.

\begin{figure}
	\centering
	\begin{subfigure}[t]{0.3\linewidth}
		\centering
		\begin{tikzpicture}[scale=0.75,baseline=(current bounding box.center)]
		%horizontal lines
		\draw (-0.25,0) -- (3.25,0);
		\draw (-0.25,-1) -- (3.25,-1);
		\draw (-0.25,-2) -- (1,-2);
		\draw (2,-2) -- (3.25,-2);
		\draw (-0.25,-3) -- (3.25,-3);
		%vertical lines
		\draw (0,-3.25) -- (0,0.25);
		\draw (1,-3.25) -- (1,0.25);
		\draw (2,-3.25) -- (2,0.25);
		\draw (3,-3.25) -- (3,0.25);
		\end{tikzpicture}
		\caption{One missing qubit defect on a toric code lattice.}\label{fig:1defect}
	\end{subfigure}\vspace{20pt}
	
	\begin{subfigure}[t]{0.3\linewidth}
		\centering
		\begin{tikzpicture}[scale=0.75,baseline=(current bounding box.center)]
		%horizontal lines
		\draw (-0.25,0) -- (3.25,0);
		\draw (-0.25,-1) -- (1,-1);
		\draw (2,-1) -- (3.25,-1);
		\draw (-0.25,-2) -- (1,-2);
		\draw (2,-2) -- (3.25,-2);
		\draw (-0.25,-3) -- (3.25,-3);
		%vertical lines
		\draw (0,-3.25) -- (0,0.25);
		\draw (1,-3.25) -- (1,0.25);
		\draw (2,-3.25) -- (2,0.25);
		\draw (3,-3.25) -- (3,0.25);
		\end{tikzpicture}
		\caption{Two adjacent missing qubit defects on a toric code lattice.}\label{fig:2adjdefects}
	\end{subfigure}\vspace{20pt}
	
	\begin{subfigure}[t]{0.3\linewidth}
		\centering
		\begin{tikzpicture}[scale=0.75,baseline=(current bounding box.center)]
		%horizontal lines
		\draw (-0.25,0) -- (3.25,0);
		\draw (-0.25,-1) -- (2,-1);
		\draw (3,-1) -- (3.25,-1);
		\draw (-0.25,-2) -- (1,-2);
		\draw (2,-2) -- (3.25,-2);
		\draw (-0.25,-3) -- (3.25,-3);
		%vertical lines
		\draw (0,-3.25) -- (0,0.25);
		\draw (1,-3.25) -- (1,0.25);
		\draw (2,-3.25) -- (2,0.25);
		\draw (3,-3.25) -- (3,0.25);
		\end{tikzpicture}
		\caption{Two distant missing qubit defects on a toric code lattice.}\label{fig:2distdefects}
	\end{subfigure}
	\caption{Depiction of different situations where missing qubit defects are inserted on a toric code lattice. In the case of more than one missing link the ground space for the system depends on whether the defects are adjacent to each other.}
\end{figure}

Writing out the whole ground eigenspace is a highly intricately combinatorial problem, which is why we exploit concepts from category theory to model defects in the system. Of course, the above description is not only restricted to the toric code case. In Sec.~\ref{Ising}, we show how this construction works for a one-dimensional spin system and we also show how ideas and techniques from category theory can be used to model dynamical defects in such a one-dimensional spin system. But for now, let us introduce the basic category theoretical toolbox needed for studying dynamical defects in those spin chains.
\input{2_2_Fusion_Category.tex}
\input{2_3_Bimodules.tex}
\input{2_4_VecZp_Bimodules.tex}
\input{2_5_Bimodule_Tensor_Products.tex}
\input{2_6_Annular_Category.tex}

\section{A $\Vec(\Z/2\Z)$ spin chain}
\input{3_1_Spin_Chain.tex}
\subsection{Computation of the bimodule vertex}\label{subsec:VecZ2}

\subsection*{Step 1: Compute isomorphism classes of objects and pick a representative.} In general, what we aim for are representations of the annular category of the form
	\begin{figure}[H]
		\centering
		\begin{tikzpicture}[scale=1.2,baseline=(current bounding box.center)]
			\draw (0,0) circle (0.5cm);
			\draw[bwred,line width=0.4mm]
			\foreach \a in {-60, -120} {
				(\a:0.5) -- (\a:1.6)
			};
			\draw ([shift=(-60:1cm)]0,0) arc (-60:90:1cm);
			\draw ([shift=(-120:1.15cm)]0,0) arc (-120:-60:1.15cm);
			\draw ([shift=(240:0.85cm)]0,0) arc (240:90:0.85cm);
			\draw[] (90:0.5) -- (90:1.6);
			\node at (90:.3cm){$a$};\node at (90:1.75cm){$a+x+z$};
			\node at (-1,0.3) {$x$};
			\node at (1.15,0.3) {$z$};
			\node at (0,-1.3) {$y$};
		\end{tikzpicture}.
	\end{figure}
\noindent
According to the figure above, $(*,*,a)\cong(*,*,a+x+z)$ for all possible labels $x,y,z$, so we pick $(*,*,0)$ as a representative.

\subsection*{Step 2: Find primitive idempotents.} To find a representation of the annular category, we need to compute the primitive idempotents, i.e. we need to find morphisms that map $(*,*,0)$ to $(*,*,0)$ which square to themselves and are orthogonal to each other. Additionally, since we want \emph{primitive} idempotents, we need to make sure that they cannot be written as the sum of idempotents.
Candidates for idempotents are the morphism where $x+z=0$:
	\begin{align}
		T_{0,0}:={}&
		\begin{tikzpicture}[scale=1,baseline=(current bounding box.center)]
			\draw (0,0) circle (0.5cm);
			\draw[bwred,line width=0.4mm]
			\foreach \a in {-60, -120} {
				(\a:0.5) -- (\a:1.6)
			};
%			\draw ([shift=(-60:1cm)]0,0) arc (-60:90:1cm);
%			\draw ([shift=(-120:1.15cm)]0,0) arc (-120:-60:1.15cm);
%			\draw ([shift=(240:0.85cm)]0,0) arc (240:90:0.85cm);
			\draw[] (90:0.5) -- (90:1.6);
			\node at (90:.3cm){$0$};\node at (90:1.75cm){$0$};
%			\node at (-1,0.3) {$0$};
%			\node at (1.15,0.3) {$0$};
%			\node at (0,-1.3) {$0$};
		\end{tikzpicture}\\
%		\hspace{5mm}
		T_{0,1}:={}&
		\begin{tikzpicture}[scale=1,baseline=(current bounding box.center)]
		\draw (0,0) circle (0.5cm);
		\draw[bwred,line width=0.4mm]
		\foreach \a in {-60, -120} {
			(\a:0.5) -- (\a:1.6)
		};
%		\draw ([shift=(-60:1cm)]0,0) arc (-60:90:1cm);
		\draw ([shift=(-120:1.15cm)]0,0) arc (-120:-60:1.15cm);
%		\draw ([shift=(240:0.85cm)]0,0) arc (240:90:0.85cm);
		\draw[] (90:0.5) -- (90:1.6);
		\node at (90:.3cm){$0$};\node at (90:1.75cm){$0$};
%		\node at (-1,0.3) {$0$};
%		\node at (1.15,0.3) {$0$};
		\node at (0,-1.3) {$1$};
		\end{tikzpicture}\\
%		\hspace{5mm}
		T_{1,0}:={}&
		\begin{tikzpicture}[scale=1,baseline=(current bounding box.center)]
		\draw (0,0) circle (0.5cm);
		\draw[bwred,line width=0.4mm]
		\foreach \a in {-60, -120} {
			(\a:0.5) -- (\a:1.6)
		};
		\draw ([shift=(-60:1cm)]0,0) arc (-60:90:1cm);
%		\draw ([shift=(-120:1.15cm)]0,0) arc (-120:-60:1.15cm);
		\draw ([shift=(240:0.85cm)]0,0) arc (240:90:0.85cm);
		\draw[] (90:0.5) -- (90:1.6);
		\node at (90:.3cm){$0$};\node at (90:1.75cm){$0$};
		\node at (-1,0.3) {$1$};
		\node at (1.15,0.3) {$1$};
%		\node at (0,-1.3) {$0$};
		\end{tikzpicture}\\
%		\hspace{5mm}
		T_{1,1}:={}&
		\begin{tikzpicture}[scale=1,baseline=(current bounding box.center)]
		\draw (0,0) circle (0.5cm);
		\draw[bwred,line width=0.4mm]
		\foreach \a in {-60, -120} {
			(\a:0.5) -- (\a:1.6)
		};
		\draw ([shift=(-60:1cm)]0,0) arc (-60:90:1cm);
		\draw ([shift=(-120:1.15cm)]0,0) arc (-120:-60:1.15cm);
		\draw ([shift=(240:0.85cm)]0,0) arc (240:90:0.85cm);
		\draw[] (90:0.5) -- (90:1.6);
		\node at (90:.3cm){$0$};\node at (90:1.75cm){$0$};
		\node at (-1,0.3) {$1$};
		\node at (1.15,0.3) {$1$};
		\node at (0,-1.3) {$1$};
		\end{tikzpicture}.
	\end{align}
\noindent
The first morphism can be interpreted as the identity morphism and it obviously squares to itself. It is easy to see that the second and third diagrams square to the first. To square the final morphism, we need to do the following calculation:\vspace{5pt}
	\begin{align*}
		\begin{tikzpicture}[scale=1,baseline=(current bounding box.center)]
			\draw (0,0) circle (0.5cm);
			\draw[bwred,line width=0.4mm]
			\foreach \a in {-60, -120} {
				(\a:0.5) -- (\a:2)
			};
			\draw[] (90:0.5) -- (90:2);
			\draw ([shift=(-60:1cm)]0,0) arc (-60:90:1cm);
			\draw ([shift=(-120:1.15cm)]0,0) arc (-120:-60:1.15cm);
			\draw ([shift=(240:0.85cm)]0,0) arc (240:90:0.85cm);
			\draw ([shift=(240:1.3cm)]0,0) arc (240:90:1.3cm);
			\draw ([shift=(-60:1.45cm)]0,0) arc (-60:90:1.45cm);
			\draw ([shift=(-120:1.6cm)]0,0) arc (-120:-60:1.6cm);
			\node at (90:.3cm){$0$};\node at (90:2.25cm){$0$};
			\node at (-1,0.3) {$1$};
			\node at (1.15,0.3) {$1$};
			\node at (0,-1.35) {$1$};
			\node at (-1.4,0.6) {$1$};
			\node at (1.5,0.6) {$1$};
			\node at (0,-1.8) {$1$};
		\end{tikzpicture}
		&=-\begin{tikzpicture}[scale=1,baseline=(current bounding box.center)]
			\draw (0,0) circle (0.5cm);
			\draw[bwred,line width=0.4mm]
			\foreach \a in {-60, -120} {
				(\a:0.5) -- (\a:2)
			};
			\draw[] (90:0.5) -- (90:2);
			\draw ([shift=(-60:1cm)]0,0) arc (-60:90:1cm);
			\draw ([shift=(-120:1.25cm)]0,0) arc (-120:-60:1.25cm);
			\draw ([shift=(240:0.85cm)]0,0) arc (240:90:0.85cm);
			\draw ([shift=(240:1.15cm)]0,0) arc (240:90:1.15cm);
%			\draw[] ([shift=(240:1.0cm)]0,0) to [bend left=80] ([shift=(90:1.4cm)]0,0);
			\draw ([shift=(-60:1.45cm)]0,0) arc (-60:90:1.45cm);
			\draw ([shift=(-120:1.6cm)]0,0) arc (-120:-60:1.6cm);
			\node at (90:.3cm){$0$};\node at (90:2.25cm){$0$};
			\node at (-1.1,0.8) {$1$};
			\node at (1.15,0.3) {$1$};
			\node at (0,-1.4) {$1$};
			\node at (-0.8,0.6) {$1$};
			\node at (1.5,0.6) {$1$};
			\node at (0,-1.8) {$1$};
		\end{tikzpicture}\\
		&=\begin{tikzpicture}[scale=1,baseline=(current bounding box.center)]
			\draw (0,0) circle (0.5cm);
			\draw[bwred,line width=0.4mm]
			\foreach \a in {-60, -120} {
				(\a:0.5) -- (\a:2)
			};
			\draw[] (90:0.5) -- (90:2);
			\draw ([shift=(240:0.75cm)]0,0) arc (240:90:0.75cm);
			%\draw ([shift=(-60:1.2cm)]0,0) arc (-60:90:1.2cm);
			\draw ([shift=(-60:1cm)]0,0) to [bend right=70] ([shift=(90:1.4cm)]0,0);
			\draw ([shift=(-120:1.5cm)]0,0) arc (-120:-60:1.5cm);
			\draw ([shift=(240:1.1cm)]0,0) arc (240:90:1.1cm);
			%\draw ([shift=(-60:1.6cm)]0,0.2) arc (-60:90:1.6cm);
			\draw ([shift=(-60:1.4cm)]0,0) to [bend right=70] ([shift=(90:1.75cm)]0,0);
			\draw ([shift=(-120:1.8cm)]0,0) arc (-120:-60:1.8cm);
			\node at (90:.3cm){$0$};\node at (90:2.25cm){$0$};
			\node at (0.2,-0.2) {$0$};
			\node at (-0.2,-0.2) {$0$};
			\node at (-1.1,-2) {$0$};
			\node at (1.1,-2) {$0$};
			\node at (-0.9,0.3) {$1$};
			\node at (1.05,0.3) {$1$};
			\node at (0,-1.65) {$1$};
			\node at (-1.2,0.6) {$1$};
			\node at (1.35,0.6) {$1$};
			\node at (0,-2) {$1$};
		\end{tikzpicture}\\
		&=\begin{tikzpicture}[scale=1,baseline=(current bounding box.center)]
			\draw (0,0) circle (0.5cm);
			\draw[bwred,line width=0.4mm]
			\foreach \a in {-60, -120} {
				(\a:0.5) -- (\a:1.6)
			};
			\draw[] (90:0.5) -- (90:1.6);
			\node at (90:.3cm){$0$};\node at (90:1.75cm){$0$};
%			\draw ([shift=(-60:1cm)]0,0) arc (-60:90:1cm);
%			\draw ([shift=(-120:1.15cm)]0,0) arc (-120:-60:1.15cm);
%			\draw ([shift=(240:0.85cm)]0,0) arc (240:90:0.85cm);
%			\node at (0.2,-0.2) {$0$};
%			\node at (-0.2,-0.2) {$0$};
%			\node at (-0.9,-1.6) {$0$};
%			\node at (0.9,-1.6) {$0$};
%			\node at (-1,0.3) {$0$};
%			\node at (1.15,0.3) {$0$};
%			\node at (0,-1.35) {$0$};
		\end{tikzpicture}\vspace{5pt}.
	\end{align*}
Hence, the second candidate does not square to itself but to the first candidate. 
Similar computations show that $T_{a,b}T_{c,d}=T_{a+c,b+d}$. 
However, since the primitive idempotents form an algebra, we can also consider linear combinations of the candidates. Also, because we have four candidates, we know that the algebra is $4$-dimensional. To find out the primitive idempotents from the candidates, it is convenient to find a representation of them in terms of matrices, i.e.\ we need matrices that multiply in the same way as the candidates. As mentioned above, the first candidate is the identity, hence the corresponding matrix is
	\begin{equation}
		M_{0,0}=\begin{pmatrix}
			1 & 0 & 0 & 0\\
			0 & 1 & 0 & 0\\
			0 & 0 & 1 & 0\\
			0 & 0 & 0 & 1\\
		\end{pmatrix}.
	\end{equation}
From this representation it is clear that this candidate is an idempotent, but not a primitive one: $M_{0,0}$ can be written as
	\begin{equation}
	\begin{split}
	\label{eq_M0}
		M_{0,0}={}&
		\begin{pmatrix}
		1 & 0 & 0 & 0\\
		0 & 0 & 0 & 0\\
		0 & 0 & 0 & 0\\
		0 & 0 & 0 & 0\\
		\end{pmatrix}
		+\begin{pmatrix}
		0 & 0 & 0 & 0\\
		0 & 1 & 0 & 0\\
		0 & 0 & 0 & 0\\
		0 & 0 & 0 & 0\\
		\end{pmatrix}\\
		&+
		\begin{pmatrix}
		0 & 0 & 0 & 0\\
		0 & 0 & 0 & 0\\
		0 & 0 & 1 & 0\\
		0 & 0 & 0 & 0\\
		\end{pmatrix}
		+\begin{pmatrix}
		0 & 0 & 0 & 0\\
		0 & 0 & 0 & 0\\
		0 & 0 & 0 & 0\\
		0 & 0 & 0 & 1\\
		\end{pmatrix}
	\end{split}		.
	\end{equation}
The matrix representing the second candidate has to fulfill $M_{0,1}^2=M_{0,0}$, so a possible choice is
	\begin{equation}
		M_{0,1}=\begin{pmatrix}
			1 & 0 & 0 & 0\\
			0 & 1 & 0 & 0\\
			0 & 0 & -1 & 0\\
			0 & 0 & 0 & -1\\
		\end{pmatrix}.
	\end{equation}
	Note that this representation is not unique. Similar matrices can be found for the remaining morphisms.
From \eqref{eq_M0}, we get four candidates for primitive idempotents. They are indeed primitive idempotents in the algebra of $4$-dimensional matrices. To translate them back into annular diagrams, we just need to express them in terms of $M_{a,b}$, which yields
	\begin{align}
	P_{x,y}=\frac{1}{4}\sum_{a,b}(-1)^{ax+by}M_{a,b}
	\end{align}
which is diagrammatically
	\begin{equation}
	\begin{split}
		P_{x,y}=\frac{1}{4}&
		\left(
			\begin{tikzpicture}[scale=1,baseline=(current bounding box.center)]
			\draw (0,0) circle (0.5cm);
			\draw[bwred,line width=0.4mm]
			\foreach \a in {-60, -120} {
				(\a:0.5) -- (\a:1.6)
			};
			%			\draw ([shift=(-60:1cm)]0,0) arc (-60:90:1cm);
			%			\draw ([shift=(-120:1.15cm)]0,0) arc (-120:-60:1.15cm);
			%			\draw ([shift=(240:0.85cm)]0,0) arc (240:90:0.85cm);
			\draw[] (90:0.5) -- (90:1.6);
			\node at (90:.3cm){$0$};\node at (90:1.75cm){$0$};
			%			\node at (-1,0.3) {$0$};
			%			\node at (1.15,0.3) {$0$};
			%			\node at (0,-1.3) {$0$};
			\end{tikzpicture}
			+(-1)^x
			\begin{tikzpicture}[scale=1,baseline=(current bounding box.center)]
			\draw (0,0) circle (0.5cm);
			\draw[bwred,line width=0.4mm]
			\foreach \a in {-60, -120} {
				(\a:0.5) -- (\a:1.6)
			};
%						\draw ([shift=(-60:1cm)]0,0) arc (-60:90:1cm);
			\draw ([shift=(-120:1.15cm)]0,0) arc (-120:-60:1.15cm);
%						\draw ([shift=(240:0.85cm)]0,0) arc (240:90:0.85cm);
			\draw[] (90:0.5) -- (90:1.6);
			\node at (90:.3cm){$0$};\node at (90:1.75cm){$0$};
			%			\node at (-1,0.3) {$0$};
%						\node at (1.15,0.3) {$0$};
						\node at (0,-1.3) {$1$};
			\end{tikzpicture}\right.\\
			&+(-1)^y
			\begin{tikzpicture}[scale=1,baseline=(current bounding box.center)]
			\draw (0,0) circle (0.5cm);
			\draw[bwred,line width=0.4mm]
			\foreach \a in {-60, -120} {
				(\a:0.5) -- (\a:1.6)
			};
									\draw ([shift=(-60:1cm)]0,0) arc (-60:90:1cm);
%			\draw ([shift=(-120:1.15cm)]0,0) arc (-120:-60:1.15cm);
									\draw ([shift=(240:0.85cm)]0,0) arc (240:90:0.85cm);
			\draw[] (90:0.5) -- (90:1.6);
			\node at (90:.3cm){$0$};\node at (90:1.75cm){$0$};
						\node at (-1,0.3) {$1$};
									\node at (1.15,0.3) {$1$};
%			\node at (0,-1.3) {$1$};
			\end{tikzpicture}\\
			&\left.+(-1)^{x+y}
			\begin{tikzpicture}[scale=1,baseline=(current bounding box.center)]
			\draw (0,0) circle (0.5cm);
			\draw[bwred,line width=0.4mm]
			\foreach \a in {-60, -120} {
				(\a:0.5) -- (\a:1.6)
			};
									\draw ([shift=(-60:1cm)]0,0) arc (-60:90:1cm);
			\draw ([shift=(-120:1.15cm)]0,0) arc (-120:-60:1.15cm);
									\draw ([shift=(240:0.85cm)]0,0) arc (240:90:0.85cm);
			\draw[] (90:0.5) -- (90:1.6);
			\node at (90:.3cm){$0$};\node at (90:1.75cm){$0$};
			\node at (-1,0.3) {$1$};
			\node at (1.15,0.3) {$1$};
			\node at (0,-1.3) {$1$};
			\end{tikzpicture}
			\right).
			\end{split}
	\end{equation}
	
\subsection*{Step 3: Check for isomorphism classes of primitive idempotents.} In general, it is possible that the primitive idempotents we just found are isomorphic to each other, which means that there are matrices within the algebra that we can multiply to $P_{0,0}$, for example, and get $P_{1,0}$. The following equation is an example for how this works:
	\begin{equation}
		\begin{pmatrix} 0 & 0 & 0 & 0\\ 1 & 0 & 0 & 0\\ 0 & 0 & 0 & 0\\ 0 & 0 & 0 & 0\\ \end{pmatrix}P_{0,0} \begin{pmatrix} 0 & 1 & 0 & 0\\ 0 & 0 & 0 & 0\\ 0 & 0 & 0 & 0\\ 0 & 0 & 0 & 0\\ \end{pmatrix}=P_{1,0}.\label{eqn:isoids}
	\end{equation}
However, the two matrices we multiply $P_{0,0}$ with are not elements of the matrix algebra (the matrices on the algebra only have entries on the diagonal). If they were elements of the algebra, they would form an isomorphism between $P_{0,0}$ and $P_{1,0}$, so we would pick either $P_{0,0}$ or $P_{1,0}$ as a representative.

This step can be much more complicated for algebras with bigger dimensions. However, there is a nice trick which helps to see how many isomorphism classes there are, which is the Artin-Wedderburn theorem (see, for example, \cite{Beachy1999}). It states that any semi-simple algebra can be decomposed into a direct sum of full matrix algebras, i.e.
	\begin{equation}
		\mathcal{M}\cong\bigoplus\mathcal{M}_d,
	\end{equation}
	where $\dim{\mathcal{M}_d}=d^2$. Within a full matrix algebra we can pick, for example, the primitive idempotent $\mathrm{diag}(1,0,\ldots)$ since the equivalent of the matrices in  \eqref{eqn:isoids} exist. We therefore only get one primitive idempotent for each full matrix algebra. In our case, the algebra is 4 dimensional. The possible decompositions are therefore 
	$\mathcal{M}\cong\mathbb{C}\oplus\mathbb{C}\oplus\mathbb{C}\oplus\mathbb{C}$ and $\mathcal{M}_2$. We found the first decomposition was correct so we get four primitive idempotents (as we showed above). 
	In case of a $5$-dimensional matrix algebra, the possible decompositions are
	\begin{equation}
		\mathcal{M}_{5-\mathrm{dim}}\cong\mathbb{C}\oplus\mathcal{M}_2(\mathbb{C}),
	\end{equation}
	or 
	\begin{equation}
		\mathcal{M}_{5-\mathrm{dim}}\cong5\mathbb{C},
	\end{equation}
so we get two or five primitive idempotents respectively.
	
\subsection*{Step 4: Build the full representation.} After we have found the primitive idempotents of the algebra, we can build the full representation. This is done by putting all possible annuli on the outside of the idempotents, hence finding all the basis vectors for this space, i.e.\ all possible vectors
	\begin{equation}
		\begin{tikzpicture}[scale=1,baseline=(current bounding box.center)]
		\draw (0,0) circle (0.5cm);
		\draw[bwred,line width=0.4mm]
		\foreach \a in {-60, -120} {
			(\a:0.5) -- (\a:1.6)
		};
		\draw ([shift=(-60:1cm)]0,0) arc (-60:90:1cm);
		\draw ([shift=(-120:1.15cm)]0,0) arc (-120:-60:1.15cm);
		\draw ([shift=(240:0.85cm)]0,0) arc (240:90:0.85cm);
		\draw[] (90:0.5) -- (90:1.6);
		\node at (90:.3cm){$0$};\node at (90:1.75cm){$\alpha+\gamma$};
		\node at (-1,0.3) {$\alpha$};
		\node at (1.15,0.3) {$\gamma$};
		\node at (0,-1.4) {$\beta$};
		\end{tikzpicture}.
	\end{equation}
%\noindent
In general, the basis vectors are determined by the choice of $\alpha,\beta$ and $\gamma$, therefore there are up to $2^3=8$ possible basis vectors for each representation. However, it is possible that some of these vectors are linearly dependent, which is the case in our example, as we will see. Putting the general annuli around the primitive idempotents $P_{x,y}$ yields
	\begin{align}
	\begin{split}
	\frac{1}{4}&
	\sum_{a,b}(-1)^{ax+by}
	\begin{tikzpicture}[scale=1,baseline=(current bounding box.center)]
	\draw (0,0) circle (0.5cm);
	\draw[bwred,line width=0.4mm]
	\foreach \a in {-60, -120} {
		(\a:0.5) -- (\a:2)
	};
	\draw[] (90:0.5) -- (90:2);
	\draw ([shift=(-60:1cm)]0,0) arc (-60:90:1cm);
	\draw ([shift=(-120:1.15cm)]0,0) arc (-120:-60:1.15cm);
	\draw ([shift=(240:0.85cm)]0,0) arc (240:90:0.85cm);
	\draw ([shift=(240:1.3cm)]0,0) arc (240:90:1.3cm);
	\draw ([shift=(-60:1.45cm)]0,0) arc (-60:90:1.45cm);
	\draw ([shift=(-120:1.6cm)]0,0) arc (-120:-60:1.6cm);
	\node at (90:.3cm){$0$};\node at (90:2.25cm){$\alpha+\gamma$};
	\node at (-1,0.3) {$b$};
	\node at (1.15,0.3) {$b$};
	\node at (0,-1.35) {$a$};
	\node at (-1.4,0.6) {$\alpha$};
	\node at (1.5,0.6) {$\gamma$};
	\node at (0,-1.8) {$\beta$};
	\end{tikzpicture}\\
	&=
	\frac{1}{4}
	\sum_{a,b}(-1)^{a(x+\alpha+\gamma)+by}
	\begin{tikzpicture}[scale=1,baseline=(current bounding box.center)]
	\draw (0,0) circle (0.5cm);
	\draw[bwred,line width=0.4mm]
	\foreach \a in {-60, -120} {
		(\a:0.5) -- (\a:1.6)
	};
	\draw ([shift=(-60:1cm)]0,0) arc (-60:90:1cm);
	\draw ([shift=(-120:1.15cm)]0,0) arc (-120:-60:1.15cm);
	\draw ([shift=(240:0.85cm)]0,0) arc (240:90:0.85cm);
	\draw[] (90:0.5) -- (90:1.6);
	\node at (90:.3cm){$0$};\node at (90:1.75cm){$\alpha+\gamma$};
	\node[rotate=65] at (-1,0.3) {$\alpha+b$};
	\node[rotate=-65] at (1.15,0.3) {$\gamma+b$};
	\node at (0,-1.3) {$\beta+a$};
	\end{tikzpicture}
	\end{split}\\
	&=
	\frac{(-1)^{(x+\alpha^\prime)\beta+y\gamma}}{4}
	\sum_{a^\prime,b^\prime}(-1)^{a^\prime(x+\alpha^\prime)+b^\prime y}
	\begin{tikzpicture}[scale=1,baseline=(current bounding box.center)]
	\draw (0,0) circle (0.5cm);
	\draw[bwred,line width=0.4mm]
	\foreach \a in {-60, -120} {
		(\a:0.5) -- (\a:1.6)
	};
	\draw ([shift=(-60:1cm)]0,0) arc (-60:90:1cm);
	\draw ([shift=(-120:1.15cm)]0,0) arc (-120:-60:1.15cm);
	\draw ([shift=(240:0.85cm)]0,0) arc (240:90:0.85cm);
	\draw[] (90:0.5) -- (90:1.6);
	\node at (90:.3cm){$0$};\node at (90:1.75cm){$\alpha^\prime$};
	\node[rotate=65] at (-1,0.3) {$\alpha^\prime+b^\prime$};
	\node at (1.15,0.3) {$b^\prime$};
	\node at (0,-1.3) {$a^\prime$};
	\end{tikzpicture}
	\end{align}
which is a vector in the morphism space
	\begin{equation}	
		\begin{tikzpicture}
			\draw[bwred,line width=0.4mm] (0,0) -- (1.5,0);
			\draw (0.75,0) -- (0.75,1) node [pos=1.25]{$\alpha^\prime$};
		\end{tikzpicture}.
	\end{equation}
We now have to find a basis for every one of those morphism spaces, i.e.\ for every choice of $\alpha^\prime$. For fixed representation (fixed, $x,y$), we find a unique vector up to a multiplicative scalar for each $\alpha^\prime$, so each morphism space is one dimensional. We define the basis to be
	\begin{equation}
		\begin{tikzpicture}[scale=1,baseline=(current bounding box.center)]
			\draw[bwred,line width=0.4mm] (0,-.25) -- (.75,0) -- (1.5,-.25);
			\draw (0.75,0) -- (0.75,1) node [pos=1.25] {$\alpha$};
			\node[above,right] at (.75,.15) {$(x,y)$};
		\end{tikzpicture}\equiv
		\frac{1}{4}
		\sum_{a,b}(-1)^{a(x+\alpha)+b y}
		\begin{tikzpicture}[scale=1,baseline=(current bounding box.center)]
		\draw (0,0) circle (0.5cm);
		\draw[bwred,line width=0.4mm]
		\foreach \a in {-60, -120} {
			(\a:0.5) -- (\a:1.6)
		};
		\draw ([shift=(-60:1cm)]0,0) arc (-60:90:1cm);
		\draw ([shift=(-120:1.15cm)]0,0) arc (-120:-60:1.15cm);
		\draw ([shift=(240:0.85cm)]0,0) arc (240:90:0.85cm);
		\draw[] (90:0.5) -- (90:1.6);
		\node at (90:.3cm){$0$};\node at (90:1.75cm){$\alpha$};
		\node[rotate=65] at (-1,0.3) {$\alpha+b$};
		\node at (1.15,0.3) {$b$};
		\node at (0,-1.3) {$a$};
		\end{tikzpicture}.
	\end{equation}
The `inflation trick' developed in \cite{BBJ18,BB19a,BB19b} picks out the representation $P_{0,0}$, and we work with that from here on. This means
	\begin{equation}
	\begin{tikzpicture}[scale=1,baseline=(current bounding box.center)]
	\draw[bwred,line width=0.4mm] (0,-.25) -- (.75,0) -- (1.5,-.25);
	\draw (0.75,0) -- (0.75,1) node [pos=1.25] {$\alpha$};
	\end{tikzpicture}:=
	\frac{1}{4}
	\sum_{a,b}(-1)^{a\alpha}
	\begin{tikzpicture}[scale=1,baseline=(current bounding box.center)]
	\draw (0,0) circle (0.5cm);
	\draw[bwred,line width=0.4mm]
	\foreach \a in {-60, -120} {
		(\a:0.5) -- (\a:1.6)
	};
	\draw ([shift=(-60:1cm)]0,0) arc (-60:90:1cm);
	\draw ([shift=(-120:1.15cm)]0,0) arc (-120:-60:1.15cm);
	\draw ([shift=(240:0.85cm)]0,0) arc (240:90:0.85cm);
	\draw[] (90:0.5) -- (90:1.6);
	\node at (90:.3cm){$0$};\node at (90:1.75cm){$\alpha$};
	\node[rotate=65] at (-1,0.3) {$\alpha+b$};
	\node at (1.15,0.3) {$b$};
	\node at (0,-1.3) {$a$};
	\end{tikzpicture}
	\end{equation}

\subsection*{Step 5: Find associator of the extended category.} After the vertex itself is defined, we want to compute the $F$-symbols related to the new object. 
Our goal is to define a Hamiltonian for the chain which will be of the form
	\begin{equation*}
		\begin{tikzpicture}[scale=1,baseline=(current bounding box.center)]
		\draw (.5,.25)--(.5,.75) node [pos=.5,left]{0};
		\draw [bwred, line width=0.25mm] (0,0) -- (.5,.25) -- (1,0);
		\draw [bwred, line width=0.25mm] (0,1) -- (.5,.75)--(1,1);
		\end{tikzpicture}
		=
		\alpha\ \begin{tikzpicture}[scale=1,baseline=(current bounding box.center)]
			\draw [bwred, line width=0.25mm] (0,0) -- (0,1);
			\draw [bwred, line width=0.25mm] (0.5,0) -- (0.5,1);
		\end{tikzpicture}
		+\beta\ \begin{tikzpicture}[scale=1,baseline=(current bounding box.center)]
			\draw (0,0.5) -- (0.5,0.5);
			\draw [bwred, line width=0.25mm] (0,0) -- (0,1);
			\draw [bwred, line width=0.25mm] (0.5,0) -- (0.5,1);			
		\end{tikzpicture}\ .
	\end{equation*}
To compute $\alpha$ and $\beta$ above, in addition to the action on the spin chain, we will need the full set of $F$-symbols.

From the category $\Vec(\Z/2\Z)$, and the bimodule associators we have
\begin{align}
\left(F_{abc}^{a+b+c}\right)_{a+b,b+c}&=1&&\text{From Example~\ref{example:vecG}:} \\& &&F=+1\nonumber\\
\left(F_{ab*}^*\right)_{a+b,*}&=1&&\text{From \eqref{eqn:L}:} L=+1 \\
\left(F_{a*b}^*\right)_{*,*}&=(-1)^{ab}&&\text{From \eqref{eqn:F1}} \\
\left(F_{*ab}^*\right)_{*,a+b}&=1&&\text{From \eqref{eqn:R}:} R=+1.
\end{align}
\noindent
However, there are still some yet unknown $F$-symbols, namely
\begin{align}
	\left(F_{a**}^{a+b}\right)_{*,b}&=??\\
	\left(F_{*a*}^b\right)_{*,*}&=??\\
	\left(F_{**a}^{a+b}\right)_{b,*}&=??\\
	\left(F_{***}^*\right)_{a,b}&=??.
\end{align}
\noindent	
To compute those, we use the following normalization \cite{Bonderson,BSS08}:
	\begin{equation}
		\begin{tikzpicture}[scale=1.2,baseline=(current bounding box.center)]
			\draw (0,0) to node [left] {$a$} (0,1);
			\draw (0.5,0) to node [right] {$b$} (0.5,1);
		\end{tikzpicture}=\sum_c\ \sqrt{\frac{d_c}{d_a d_b}}\ 
		\begin{tikzpicture}[scale=1.2,baseline=(current bounding box.center)]
			\draw (0,0) to node [left] {$a$} (0.25,0.25);
			\draw (0.5,0) to node [right] {$b$} (0.25,0.25);
			\draw (0,1) to node [left] {$a$} (0.25,0.75);
			\draw (0.5,1) to node [right] {$b$} (0.25,0.75);
			\draw (0.25,0.25) to node [right] {$c$} (0.25,0.75);
		\end{tikzpicture}.
	\end{equation}
For the black strings in our diagrams, the sum has only one term and the coefficients are $d_0=1=d_1$, which yields the relation
	\begin{equation}
	\label{eq:completeness}
		\begin{tikzpicture}[scale=1.2,baseline=(current bounding box.center)]
			\draw (0,0) to node [left] {$a$} (0,1);
			\draw (0.5,0) to node [right] {$b$} (0.5,1);
		\end{tikzpicture}=
		\begin{tikzpicture}[scale=1.2,baseline=(current bounding box.center)]
			\draw (0,0) to node [left] {$a$} (0.25,0.25);
			\draw (0.5,0) to node [right] {$b$} (0.25,0.25);
			\draw (0,1) to node [left] {$a$} (0.25,0.75);
			\draw (0.5,1) to node [right] {$b$} (0.25,0.75);
			\draw (0.25,0.25) to node [right] {$a+b$} (0.25,0.75);
		\end{tikzpicture}.
	\end{equation}
From the fusion rules, $d_*=\sqrt{2}$, so
\begin{equation}
\label{eq:completeness2}
\begin{tikzpicture}[scale=1.2,baseline=(current bounding box.center)]
\draw[bwred,line width=0.4mm] (0,0) to (0,1);
\draw (0.5,0) to node [right] {$a$} (0.5,1);
\end{tikzpicture}=
\begin{tikzpicture}[scale=1.2,baseline=(current bounding box.center)]
\draw[bwred,line width=0.4mm] (0,0) to (0.25,0.25);
\draw (0.5,0) to node [right] {$a$} (0.25,0.25);
\draw[bwred,line width=0.4mm] (0,1) to (0.25,0.75);
\draw (0.5,1) to node [right] {$a$} (0.25,0.75);
\draw[bwred,line width=0.4mm] (0.25,0.25) to (0.25,0.75);
\end{tikzpicture}.
\end{equation}
The computation for $F_{**a}^{a+b}$ is particularly straightforward:
	\begin{align}
		\begin{tikzpicture}[scale=0.7,baseline=(current bounding box.center)]
			\draw[] (0,0) -- (0,1) node[pos=1,above]{$a+b$};
			\draw[] (0,0) -- (-0.707107,-0.707107) node [pos=.5,left] {$b$};
			\draw[bwred,line width=0.4mm] (-0.707107,-0.707107) -- (-1.41421,-1.41421);
			\draw[bwred,line width=0.4mm] (-0.707107,-0.707107) -- (0,-1.41421);
			\draw (0,0) -- (0.707107,-0.707107);
			\draw (0.707107,-0.707107) -- (1.41421,-1.41421);
			\node at (-1.41421,-1.7) {$*$};
			\node at (0,-1.7) {$*$};
			\node at (1.41421,-1.7) {$a$};
		\end{tikzpicture}&=
		\frac{1}{4}\sum_{x,y}(-1)^{bx}
		\begin{tikzpicture}[scale=.75,baseline=(current bounding box.center)]
			\draw (0,0) circle (0.5cm);
			\draw[bwred,line width=0.4mm]
			\foreach \a in {-60, -120} {
				(\a:0.5) -- (\a:1.6)
			};
			\draw[] (90:0.5) -- (90:1.6);
			\draw ([shift=(-60:1cm)]0,0) arc (-60:90:1cm);
			\draw ([shift=(-120:1.15cm)]0,0) arc (-120:-60:1.15cm);
			\draw ([shift=(240:0.85cm)]0,0) arc (240:90:0.85cm);
			\draw[] (0,1.6) -- (0.9,2.1) node[pos=0,left] {$b$};
			\draw[] (0.9,2.1) -- (0.9,2.6);
			\draw (0.9,2.1) -- (2.3,-1.35);
			\node[rotate=60] at (-1,0.3) {$b+y$};
			\node at (1.15,0.3) {$y$};
			\node at (2.3,-1.7) {$a$};
			\node at (0,-1.35) {$x$};
		\end{tikzpicture}
		\\
		=\frac{1}{4}\sum_{x,y}&(-1)^{bx}
		\begin{tikzpicture}[scale=.75,baseline=(current bounding box.center)]
		\draw (0,0) circle (0.5cm);
		\draw[bwred,line width=0.4mm]
		\foreach \a in {-60, -120} {
			(\a:0.5) -- (\a:2.5)
		};
		\draw[] (90:0.5) -- (90:1.6);
		\draw ([shift=(-60:1cm)]0,0) arc (-60:90:1cm);
		\draw ([shift=(-120:1.15cm)]0,0) arc (-120:-60:1.15cm);
		\draw ([shift=(240:0.85cm)]0,0) arc (240:90:0.85cm);
		\draw[] (0,1.6) -- (0.9,2.1) node[pos=0,left] {$b$};
		\draw[] (0.9,2.1) -- (0.9,2.6);
%		\draw (0.9,2.1) -- (2.3,-1.35);
		\draw (-60:1.6) -- (2.3,-2) node[pos=1,below] {$a$};
		\draw (-60:1.4) --(1.5,0) node[right]{$a$} -- (0.9,2.1);
		\node[rotate=60] at (-1,0.3) {$b+y$};
		\node at (1.15,0.3) {$y$};
%		\node at (2.3,-1.7) {$a$};
		\node at (0,-1.35) {$x$};
		%\draw[bwred,line width=0.4mm] (-60:1.6)--(-60:2.5);
		\end{tikzpicture}
		\\
		=\frac{1}{4}\sum_{x,y}&(-1)^{(a+b)x}
		\begin{tikzpicture}[scale=.75,baseline=(current bounding box.center)]
		\draw (0,0) circle (0.5cm);
		\draw[bwred,line width=0.4mm]
		\foreach \a in {-60, -120} {
			(\a:0.5) -- (\a:2.5)
		};
		\draw[] (90:0.5) -- (90:1.6);
		\draw ([shift=(-60:1cm)]0,0) arc (-60:90:1cm);
		\draw ([shift=(-120:1.5cm)]0,0) arc (-120:-60:1.5cm);
		\draw ([shift=(240:0.85cm)]0,0) arc (240:90:0.85cm);
		\draw[] (0,1.6) -- (0.9,2.1) node[pos=0,left] {$b$};
		\draw[] (0.9,2.1) -- (0.9,2.6);
		%		\draw (0.9,2.1) -- (2.3,-1.35);
		\draw (-60:1.6) -- (2.3,-2) node[pos=1,below] {$a$};
		\draw (-60:1.4) --(1.5,0) node[right]{$a$} -- (0.9,2.1);
		\node[rotate=60] at (-1,0.3) {$b+y$};
		\node at (1.15,0.3) {$y$};
		%		\node at (2.3,-1.7) {$a$};
		\node at (0,-1.35) {$x$};
		%\draw[bwred,line width=0.4mm] (-60:1.6)--(-60:2.5);
		\end{tikzpicture}
		\\
		=\frac{1}{4}\sum_{x,y}&(-1)^{(a+b)x}
		\begin{tikzpicture}[scale=.75,baseline=(current bounding box.center)]
		\draw (0,0) circle (0.5cm);
		\draw[bwred,line width=0.4mm]
		\foreach \a in {-60, -120} {
			(\a:0.5) -- (\a:2.5)
		};
		\draw[] (90:0.5) -- (90:1.6) node[pos=1,above]{$a+b$};
		\draw ([shift=(-60:1cm)]0,0) arc (-60:90:1cm);
		\draw ([shift=(-120:1.5cm)]0,0) arc (-120:-60:1.5cm);
		\draw ([shift=(240:0.85cm)]0,0) arc (240:90:0.85cm);
%		\draw[] (0,1.6) -- (0.9,2.1) ;
%		\draw[] (0.9,2.1) -- (0.9,2.6);
		%		\draw (0.9,2.1) -- (2.3,-1.35);
		\draw (-60:1.6) -- (2.3,-2) node[pos=1,below] {$a$};
%		\draw (-60:1.4) --(1.5,0) node[right]{$a$} -- (0.9,2.1);
		\node[rotate=60] at (-1,0.3) {$b+y$};
		\node[rotate=-60] at (1.15,0.3) {$y+a$};
		%		\node at (2.3,-1.7) {$a$};
		\node at (0,-1.35) {$x$};
		%\draw[bwred,line width=0.4mm] (-60:1.6)--(-60:2.5);
		\end{tikzpicture}
		\\
		=\frac{1}{4}\sum_{x,y^\prime}&(-1)^{(a+b)x}
		\begin{tikzpicture}[scale=.75,baseline=(current bounding box.center)]
		\draw (0,0) circle (0.5cm);
		\draw[bwred,line width=0.4mm]
		\foreach \a in {-60, -120} {
			(\a:0.5) -- (\a:2.5)
		};
		\draw[] (90:0.5) -- (90:1.6) node[pos=1,above]{$a+b$};
		\draw ([shift=(-60:1cm)]0,0) arc (-60:90:1cm);
		\draw ([shift=(-120:1.5cm)]0,0) arc (-120:-60:1.5cm);
		\draw ([shift=(240:0.85cm)]0,0) arc (240:90:0.85cm);
		\draw (-60:1.6) -- (2.3,-2) node[pos=1,below] {$a$};
		\node[rotate=60] at (-1,0.3) {$a+b+y^\prime$};
		\node at (1.15,0.3) {$y^\prime$};
		\node at (0,-1.35) {$x$};
		%\draw[bwred,line width=0.4mm] (-60:1.6)--(-60:2.5);
		\end{tikzpicture}
		\\
		&=\begin{tikzpicture}[xscale=-1,scale=0.7,baseline=(current bounding box.center)]
		\draw[] (0,0) -- (0,1) node[pos=1,above]{$a+b$};
		\draw[bwred,line width=0.4mm] (0,0) -- (-0.707107,-0.707107);
		\draw[] (-0.707107,-0.707107) -- (-1.41421,-1.41421);
		\draw[bwred,line width=0.4mm] (-0.707107,-0.707107) -- (0,-1.41421);
		\draw[bwred,line width=0.4mm] (0,0) -- (0.707107,-0.707107);
		\draw[bwred,line width=0.4mm] (0.707107,-0.707107) -- (1.41421,-1.41421);
		\node at (-1.41421,-1.7) {$a$};
		\node at (0,-1.7) {$*$};
		\node at (1.41421,-1.7) {$*$};
		\end{tikzpicture}
	\end{align}
so it follows that 
	\begin{equation}
		\left(F_{**a}^{a+b}\right)_{b,*}=1.
	\end{equation}
The computation of $\left(F_{a**}^{a+b}\right)_{*,b}=1$ is identical. A similar computation yields $\left(F_{*a*}^{b}\right)_{*,*}=(-1)^{ab}$.

Attempting to use \eqref{eq:completeness} to compute $F_{***}^*$ is circular, so we need a different technique. There is an action of a 4-string annular category on the fusion trees on both sides of the equation
\begin{equation}
\begin{split}
\begin{tikzpicture}[scale=0.7,baseline=(current bounding box.center)]
\draw[bwred,line width=0.4mm] (0,0) -- (0,1) node[pos=1,above,text=black]{$*$};
\draw[] (0,0) -- (-0.707107,-0.707107) node [pos=.5,left] {$a$};
\draw[bwred,line width=0.4mm] (-0.707107,-0.707107) -- (-1.41421,-1.41421);
\draw[bwred,line width=0.4mm] (-0.707107,-0.707107) -- (0,-1.41421);
\draw[bwred,line width=0.4mm] (0,0) -- (0.707107,-0.707107);
\draw[bwred,line width=0.4mm] (0.707107,-0.707107) -- (1.41421,-1.41421);
\node at (-1.41421,-1.7) {$*$};
\node at (0,-1.7) {$*$};
\node at (1.41421,-1.7) {$*$};
\end{tikzpicture}
=&
\left(F_{***}^*\right)_{a,0}
\begin{tikzpicture}[xscale=-1,scale=0.7,baseline=(current bounding box.center)]
\draw[bwred,line width=0.4mm] (0,0) -- (0,1) node[pos=1,above,text=black]{$*$};
\draw[] (0,0) -- (-0.707107,-0.707107) node [pos=.5,left] {$0$};
\draw[bwred,line width=0.4mm] (-0.707107,-0.707107) -- (-1.41421,-1.41421);
\draw[bwred,line width=0.4mm] (-0.707107,-0.707107) -- (0,-1.41421);
\draw[bwred,line width=0.4mm] (0,0) -- (0.707107,-0.707107);
\draw[bwred,line width=0.4mm] (0.707107,-0.707107) -- (1.41421,-1.41421);
\node at (-1.41421,-1.7) {$*$};
\node at (0,-1.7) {$*$};
\node at (1.41421,-1.7) {$*$};
\end{tikzpicture}\\
&+
\left(F_{***}^*\right)_{a,1}
\begin{tikzpicture}[xscale=-1,scale=0.7,baseline=(current bounding box.center)]
\draw[bwred] (0,0) -- (0,1) node[pos=1,above,text=black]{$*$};
\draw[] (0,0) -- (-0.707107,-0.707107) node [pos=.5,left] {$1$};
\draw[bwred,line width=0.4mm] (-0.707107,-0.707107) -- (-1.41421,-1.41421);
\draw[bwred,line width=0.4mm] (-0.707107,-0.707107) -- (0,-1.41421);
\draw[bwred,line width=0.4mm] (0,0) -- (0.707107,-0.707107);
\draw[bwred,line width=0.4mm] (0.707107,-0.707107) -- (1.41421,-1.41421);
\node at (-1.41421,-1.7) {$*$};
\node at (0,-1.7) {$*$};
\node at (1.41421,-1.7) {$*$};
\end{tikzpicture}.
\end{split}
\end{equation}
Both sides of the equation transform in the same representation of this 4-string action. We use this to find the $F$-symbols. Consider applying the annulus
\begin{equation}
\begin{tikzpicture}[scale=1,baseline=(current bounding box.center)]
\draw[bwred,line width=0.4mm](-.9,0)--(-.9,-1);
\draw[bwred,line width=0.4mm](0,0)--(0,-1);\draw[bwred,line width=0.4mm](0,.2)--(0,1);
\draw[bwred,line width=0.4mm](.9,0)--(.9,-1);
\draw[rounded corners] (0,.7)--(-1.6,0)--(-.9,-.5)node[pos=.5,left]{$x_0$};
\draw[rounded corners] (0,.9)--(1.6,0)--(.9,-.5)node[pos=.5,right]{$\ x_1$};
\draw[rounded corners] (-.9,-.9)--(0,-.6)node[pos=.5,above]{$x_2$};
\draw[rounded corners] (0,-.9)--(.9,-.6)node[pos=.5,above]{$x_3$};
\draw[rounded corners] (-2,-1) rectangle (2,1);
\filldraw[fill=white,rounded corners] (-1,-.2) rectangle (1,.2);
\end{tikzpicture}\label{eqn:Ftransform}
\end{equation}
to both sides, resolving the vertices where necessary. The equation becomes
\begin{align}
(-&1)^{(a+x_0)(x_1+x_2)}
\begin{tikzpicture}[scale=0.6,baseline=(current bounding box.center)]
\draw[bwred,line width=0.4mm] (0,0) -- (0,1) node[pos=1,above,text=black]{$*$};
\draw[] (0,0) -- (-0.707107,-0.707107) node [pos=.5,left] {$a+x_0+x_3$};
\draw[bwred,line width=0.4mm] (-0.707107,-0.707107) -- (-1.41421,-1.41421);
\draw[bwred,line width=0.4mm] (-0.707107,-0.707107) -- (0,-1.41421);
\draw[bwred,line width=0.4mm] (0,0) -- (0.707107,-0.707107);
\draw[bwred,line width=0.4mm] (0.707107,-0.707107) -- (1.41421,-1.41421);
\node at (-1.41421,-1.7) {$*$};
\node at (0,-1.7) {$*$};
\node at (1.41421,-1.7) {$*$};
\end{tikzpicture}\\[10pt]
%
\begin{split}
=&
\left(F_{***}^*\right)_{a,0}(-1)^{(x_1+x_2)x_3}
\begin{tikzpicture}[xscale=-1,scale=0.6,baseline=(current bounding box.center)]
\draw[bwred,line width=0.4mm] (0,0) -- (0,1) node[pos=1,above,text=black]{$*$};
\draw[] (0,0) -- (-0.707107,-0.707107) node [pos=.5,right] {$x_1+x_2$};
\draw[bwred,line width=0.4mm] (-0.707107,-0.707107) -- (-1.41421,-1.41421);
\draw[bwred,line width=0.4mm] (-0.707107,-0.707107) -- (0,-1.41421);
\draw[bwred,line width=0.4mm] (0,0) -- (0.707107,-0.707107);
\draw[bwred,line width=0.4mm] (0.707107,-0.707107) -- (1.41421,-1.41421);
\node at (-1.41421,-1.7) {$*$};
\node at (0,-1.7) {$*$};
\node at (1.41421,-1.7) {$*$};
\end{tikzpicture}
\\&+
\left(F_{***}^*\right)_{a,1}(-1)^{x_0+(1+x_1+x_2)x_3}
\begin{tikzpicture}[xscale=-1,scale=0.6,baseline=(current bounding box.center)]
\draw[bwred,line width=0.4mm] (0,0) -- (0,1) node[pos=1,above,text=black]{$*$};
\draw[] (0,0) -- (-0.707107,-0.707107) node [pos=.5,right] {$1+x_1+x_2$};
\draw[bwred,line width=0.4mm] (-0.707107,-0.707107) -- (-1.41421,-1.41421);
\draw[bwred,line width=0.4mm] (-0.707107,-0.707107) -- (0,-1.41421);
\draw[bwred,line width=0.4mm] (0,0) -- (0.707107,-0.707107);
\draw[bwred,line width=0.4mm] (0.707107,-0.707107) -- (1.41421,-1.41421);
\node at (-1.41421,-1.7) {$*$};
\node at (0,-1.7) {$*$};
\node at (1.41421,-1.7) {$*$};
\end{tikzpicture}
\end{split}
\\[10pt]
%
\begin{split}
=&(-1)^{(a+x_0)(x_1+x_2)}
	\left(
	\left(F_{***}^*\right)_{a+x_0+x_3,x_1+x_2}
	\begin{tikzpicture}[xscale=-1,scale=0.6,baseline=(current bounding box.center)]
	\draw[bwred,line width=0.4mm] (0,0) -- (0,1) node[pos=1,above,text=black]{$*$};
	\draw[] (0,0) -- (-0.707107,-0.707107) node [pos=.5,right] {$x_1+x_2$};
	\draw[bwred,line width=0.4mm] (-0.707107,-0.707107) -- (-1.41421,-1.41421);
	\draw[bwred,line width=0.4mm] (-0.707107,-0.707107) -- (0,-1.41421);
	\draw[bwred,line width=0.4mm] (0,0) -- (0.707107,-0.707107);
	\draw[bwred,line width=0.4mm] (0.707107,-0.707107) -- (1.41421,-1.41421);
	\node at (-1.41421,-1.7) {$*$};
	\node at (0,-1.7) {$*$};
	\node at (1.41421,-1.7) {$*$};
	\end{tikzpicture}
	\right.
	\\&+
	\left(F_{***}^*\right)_{a+x_0+x_3,1+x_1+x_2}\left.
	\begin{tikzpicture}[xscale=-1,scale=0.6,baseline=(current bounding box.center)]
	\draw[bwred,line width=0.4mm] (0,0) -- (0,1) node[pos=1,above,text=black]{$*$};
	\draw[] (0,0) -- (-0.707107,-0.707107) node [pos=.5,right] {$1+x_1+x_2$};
	\draw[bwred,line width=0.4mm] (-0.707107,-0.707107) -- (-1.41421,-1.41421);
	\draw[bwred,line width=0.4mm] (-0.707107,-0.707107) -- (0,-1.41421);
	\draw[bwred,line width=0.4mm] (0,0) -- (0.707107,-0.707107);
	\draw[bwred,line width=0.4mm] (0.707107,-0.707107) -- (1.41421,-1.41421);
	\node at (-1.41421,-1.7) {$*$};
	\node at (0,-1.7) {$*$};
	\node at (1.41421,-1.7) {$*$};
	\end{tikzpicture}\right)
\end{split}
\end{align}
where in the second step, \eqref{eqn:Ftransform} has been applied to the left-hand side. This equation reduces to
\begin{align}
\left(F_{***}^*\right)_{a,b}&=(-1)^{ab}\left(F_{***}^*\right)_{0,0}.
\end{align}
We cannot fix $\left(F_{***}^*\right)_{0,0}$ by simply asking which representation the vectors transform under, since there is a global rescaling freedom. Our normalization \eqref{eq:completeness} fixes 
\begin{align}
\left(F_{***}^*\right)_{a,b}&=(-1)^{ab}\frac{\varkappa_*}{\sqrt{2}},
\end{align}
where $\varkappa_*=\pm1$ is the \emph{Frobenius-Schur} indicator of $*$.



\subsection*{Step 6: Check if $F$-symbols obey the pentagon equation.} 
%To check whether the maps that we have just defined obey the pentagon equation, we need the full set of $F$-symbols. This means that, in addition to the vertex we have just computed, we also need to compute the vertex
%	\begin{figure}[H]
%		\begin{tikzpicture}
%			\draw[red,line width=0.4mm] (0,0) -- (1.5,0);
%			\draw (0.75,0) -- (0.75,1);
%		\end{tikzpicture}
%	\end{figure}
%\noindent
%Doing the same steps as above yields
%	\begin{equation*}
%		\begin{tikzpicture}[scale=1,baseline=(current bounding box.center)]
%			\draw[red,line width=0.4mm] (0,0) -- (1.5,0);
%			\draw (0.75,0) to node[near end, right] {$\mu$} (0.75,1);
%		\end{tikzpicture}=\frac{1}{4}\sum_{x,y}(-1)^{\mu y}
%		\begin{tikzpicture}[scale=1,baseline=(current bounding box.center)]
%			\draw (0,0) circle (0.5cm);
%			\draw[red,line width=0.4mm]
%			\foreach \a in {-60, -120} {
%				(\a:0.5) -- (\a:1.6)
%			};
%			\draw (90:0.5) to node[near end, right] {$\mu$} (90:1.6);
%			\draw ([shift=(-60:1cm)]0,0) arc (-60:90:1cm);
%			\draw ([shift=(-120:1.15cm)]0,0) arc (-120:-60:1.15cm);
%			\draw ([shift=(240:0.85cm)]0,0) arc (240:90:0.85cm);
%			\node at (0.2,-0.2) {$0$};
%			\node at (-0.2,-0.2) {$0$};
%			\node at (-1,0.3) {$x$};
%			\node at (1.15,0.3) {$x$};
%			\node at (0,-1.35) {$y$};
%		\end{tikzpicture}.
%	\end{equation*}
We now have the full set of $F$-symbols
	\begin{align*}
		\left(F_{abc}^{a+b+c}\right)_{a+b,b+c}&=1\\
		\left(F_{ab*}^*\right)_{a+b,*}&=1\\
		\left(F_{a*b}^*\right)_{*,*}&=(-1)^{ab}\\
		\left(F_{*ab}^*\right)_{*,a+b}&=1\\
		\left(F_{a**}^{a+b}\right)_{*,b}&=1\\
		\left(F_{*a*}^b\right)_{*,*}&=(-1)^{ab}\\
		\left(F_{**a}^{a+b}\right)_{b,*}&=1\\
		\left(F_{***}^*\right)_{a,b}&=\frac{(-1)^{ab}\varkappa_*}{\sqrt{2}},
	\end{align*}
which happen to be the $F$-symbols for the Ising category. 
\input{3_3_Hamiltonian.tex}


\section{Conclusion}\label{Conclusion}
In this paper, we have presented an exhaustive case study of a dynamical theory for defects in a one-dimensional spin chain with $\Vec(\Z/2\Z)$ fusion rules. The framework we establish can easily be extended to make sense of defect gauging in more general quantum (spin) systems. Whereas the kinematical theory can be built from distinguishable Fock space, dynamics are generally harder to implement. In order to perform this task we use a description via the ground eigenspace of a specific Hamiltonian. This reduces the problem to a combinatorial one, which can be handled by techniques from annular categories.

In our particular example of a $\Vec(\Z/2\Z)$ spin chain, allowing the presence of defects and fixing the boundary results in the Hamiltonian of the transverse Ising model and, hence, gives rise to a critical theory.

Since our approach is not only restricted to one-dimensional systems it would be interesting to study higher dimensional models, such as the Toric code, and add defects in this more general situation. Another interesting setting is to insert fusion categories with different fusion rules than $\Vec(\Z/2\Z)$ (e.g.\ the Fibonacci category) to our method. Even in this case, it provides a framework to build physical models with dynamically evolving defects. These theories can describe topological phases with more computational power than their counterparts without defects and are, therefore, more convenient for quantum information processing tasks.



\input acknowledgement.tex   % input acknowledgement


%\bibliographystyle{apsrmp4-2}
\bibliography{literature}

\end{document}
%
% ****** End of file template.aps ******
